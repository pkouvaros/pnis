%%%% ijcai24.tex

\typeout{Appendix for Paper \#6775}

% These are the instructions for authors for IJCAI-23.

\documentclass{article}
\pdfpagewidth=8.5in
\pdfpageheight=11in

% The file ijcai23.sty is a copy from ijcai22.sty
% The file ijcai22.sty is NOT the same as previous years'
\usepackage{ijcai23}

\usepackage[T1]{fontenc}
\usepackage{amsmath,amssymb,mathtools}

% Use the postscript times font!
\usepackage{times}
\usepackage{soul}
\usepackage{url}
\usepackage[hidelinks]{hyperref}
\usepackage[utf8]{inputenc}
\usepackage[small]{caption}
\usepackage{graphicx}
\usepackage{amsmath}
\usepackage{amsthm}
\usepackage{booktabs}
\usepackage{colortbl}
\usepackage{algorithm}
\usepackage{algorithmic}
\usepackage{xspace}
\usepackage{paralist}
\usepackage[switch]{lineno}
\usepackage{tikz}
\usetikzlibrary{shapes}
\usepackage{marginnote}

% Comment out this line in the camera-ready submission
% \linenumbers

\urlstyle{same}

% the following package is optional:
%\usepackage{latexsym}

% See https://www.overleaf.com/learn/latex/theorems_and_proofs
% for a nice explanation of how to define new theorems, but keep
% in mind that the amsthm package is already included in this
% template and that you must *not* alter the styling.
\newtheorem{example}{Example}
\newtheorem{theorem}{Theorem}
\newtheorem{definition}{Definition}
\newtheorem{lemma}{Lemma}
\newtheorem{corollary}{Corollary}
\newtheorem{innercustomthm}{Theorem}
\newenvironment{customthm}[1]
  {\renewcommand\theinnercustomthm{#1}\innercustomthm}
  {\endinnercustomthm}
\newtheorem{innercustomlemma}{Lemma}
\newenvironment{customlemma}[1]
  {\renewcommand\theinnercustomlemma{#1}\innercustomlemma}
  {\endinnercustomlemma}


% Following comment is from ijcai97-submit.tex:
% The preparation of these files was supported by Schlumberger Palo Alto
% Research, AT\&T Bell Laboratories, and Morgan Kaufmann Publishers.
% Shirley Jowell, of Morgan Kaufmann Publishers, and Peter F.
% Patel-Schneider, of AT\&T Bell Laboratories collaborated on their
% preparation.

% These instructions can be modified and used in other conferences as long
% as credit to the authors and supporting agencies is retained, this notice
% is not changed, and further modification or reuse is not restricted.
% Neither Shirley Jowell nor Peter F. Patel-Schneider can be listed as
% contacts for providing assistance without their prior permission.

% To use for other conferences, change references to files and the
% conference appropriate and use other authors, contacts, publishers, and
% organizations.
% Also change the deadline and address for returning papers and the length and
% page charge instructions.
% Put where the files are available in the appropriate places.


% PDF Info Is REQUIRED.
% Please **do not** include Title and Author information
\pdfinfo{
/TemplateVersion (IJCAI.2023.0)
}

\title{Appendix for Paper \#6775}

% Single author syntax
\author{
    %Panagiotis Kouvaros
    %\affiliations
    %Imperial College London, London, UK
    %\emails
    %p.kouvaros@imperial.ac.uk
}

% Multiple author syntax (remove the single-author syntax above and the \iffalse ... \fi here)
\iffalse
\author{
First Author$^1$
\and
Second Author$^2$\and
Third Author$^{2,3}$\And
Fourth Author$^4$
\affiliations
$^1$First Affiliation\\
$^2$Second Affiliation\\
$^3$Third Affiliation\\
$^4$Fourth Affiliation
\emails
\{first, second\}@example.com,
third@other.example.com,
fourth@example.com
}
\fi

\newcommand{\tuple}[1]{\left ( #1  \right )}
\newcommand{\set}[1]{\left \{ #1  \right \}}
\newcommand{\ags}[0]{\it{Ag}}
\newcommand{\ag}[1]{{#1}}
\newcommand{\lstates}[1]{L_{#1}}
\newcommand{\lstate}[1]{l_{#1}}
\newcommand{\init}[1]{\iota_{#1}}
\newcommand{\plstate}[1]{l'_{#1}}
\newcommand{\pplstate}[1]{l''_{#1}}
\newcommand{\prvs}[1]{\it{Prv}_{#1}}
\newcommand{\pers}[1]{\it{Per}_{#1}}
\newcommand{\prv}[1]{\it{prv}_{#1}}
\newcommand{\per}[1]{\it{per}_{#1}}
\newcommand{\pprv}[1]{\it{prv}'_{#1}}
\newcommand{\pper}[1]{\it{per}'_{#1}}
\newcommand{\prot}[1]{\it{prot}_{#1}}
\newcommand{\obs}[1]{\it{obs}_{#1}}
\newcommand{\tr}[1]{\it{tr}_{#1}}
\newcommand{\sys}[1]{\mathcal{S}^{(#1)}}
\newcommand{\asys}[1]{\mathcal{S}_{ab}^{(#1)}}
\newcommand{\npis}{\mathcal{S}}
\newcommand{\agents}[1]{\it{Ag}^{(n)}}
\newcommand{\ls}[2]{\it{ls}_{#1}(#2)}
\newcommand{\lprv}[2]{\it{lprv}_{#1}(#2)}
\newcommand{\lper}[2]{\it{lper}_{#1}(#2)}
\newcommand{\globalacts}[1]{\it{ACT}^{(#1)}}
\newcommand{\aglobalacts}[1]{\it{ACT}^{(#1)}_{ab}}
\newcommand{\acts}[1]{\it{Act}_{#1}}
\newcommand{\act}[1]{\it{\alpha}_{#1}}
\newcommand{\la}[2]{\it{la}_{#1}({#2})}
\newcommand{\msys}[1]{\mathcal{M}_{\sys{#1}}}
\newcommand{\masys}[1]{\mathcal{M}_{\mathcal{S}_{ab}^{(#1)}}}
\newcommand{\globalstates}[1]{G^{(#1)}}
\newcommand{\globalinit}[1]{\iota^{({#1})}}
\newcommand{\globaltr}[1]{\it{tr}^{(#1)}}
\newcommand{\globalrel}[1]{T^{(#1)}}
\newcommand{\valuation}[1]{\ell^{(#1)}}
\newcommand{\aglobalstates}[1]{G^{(#1)}_{ab}}
\newcommand{\aglobaltr}[1]{\it{tr}^{(#1)}_{ab}}
\newcommand{\aglobalrel}[1]{T^{(#1)}_{ab}}
\newcommand{\aglobalinit}[1]{\iota^{({#1})}_{ab}}
\newcommand{\avaluation}[1]{\ell^{(#1)}_{ab}}
\newcommand{\tlabel}[1]{\ell_{#1}}
\newcommand{\prop}[0]{\it{AP}}
\newcommand{\ctl}[0]{\sf{CTL}}
\newcommand{\bctl}[0]{\sf{bCTL}}
\newcommand{\bictl}[0]{\sf{bICTL}}
\newcommand{\bictlspec}[0]{\forall_{v1_,\ldots,v_m} \varphi}
\newcommand{\atprop}{\it{AP}}
\newcommand{\atvar}{\it{VAR}}
% \newcommand{\paths}[1]{\Pi_{\sys n}(#1)}
% \newcommand{\apaths}[2]{\Pi_{\asys m}^{#2}(#1)}
\newcommand{\paths}[1]{\Pi(#1)}
\newcommand{\apaths}[2]{\Pi^{#2}(#1)}
\newcommand{\pathstates}[2]{\it{States}(\Pi^{#2}(#1))}

\newcommand{\pnis}{\mathcal S}
\newcommand{\apnis}{\overline{\mathcal S}}
\newcommand{\astates}{\overline{G}}
\newcommand{\aact}{\overline{ACT}}
\newcommand{\atr}{\overline{\it{tr}}}
\newcommand{\alabel}{\overline{{\ell}}}
\newcommand{\abstate}{q_{ab}}
\newcommand{\abstatep}{q_{ab}'}
\newcommand{\abaction}{\alpha_{ab}}
\newcommand{\abactionp}{\alpha_{ab}'}
\newcommand{\abpath}{\rho_{ab}}
\newcommand{\td}[1]{\it{td}(#1)}
\newcommand{\projection}[2]{#1_{\rightarrow #2}}





\begin{document}

\maketitle


\section{Introduction}

This Appendix includes the full definition of the abstact global transition
function, the full proofs of Lemma~1 and Theorems~2,~3,~4, and furher details
pertaining to the training of the neural networks used in the evaluation
section.

\section{Proof of Lemma 1} 
We prove the symmetry reduction Lemma which we restate here for convenience.
\begin{customlemma}{1}
$\pnis \models \bictlspec$ iff $\pnis \models \varphi[v_1 \mapsto 1, \ldots, v_m
\mapsto m]$.
\end{customlemma}
\begin{proof}
Let  $n \geq m$ be arbitrary. We show that $\msys n
\models \bictlspec$ iff $\msys n \models \varphi[v_1 \mapsto 1, \ldots, v_m
\mapsto m]$. The Lemma then follows.

For the left to right direction assume that $\msys n \models \bictlspec$. Then,
    by the definition of satisfaction of $\bictl$, we have that for all $h:
    \set{v_1, \ldots, v_m} \rightarrow \set{1,\ldots,n}$,  it holds that $\msys
    n \models \varphi[v_1 \mapsto h(v_1), \ldots, v_m \mapsto h(v_m)]$.  Hence,
    $\msys n \models \varphi[v_1 \mapsto 1, \ldots, v_m \mapsto m]$.

For the right to left direction assume that
    $\msys n \models \varphi[v_1 \mapsto 1, \ldots, v_m \mapsto m]$. Let $I =
    \set{i_1, \ldots, i_m} \in \set{1, \ldots, n}^m$ be an arbitrary set of $m$
    distinct integers within $\set{1,\ldots,n}$.  Consider $\sigma : \set{1,
    \ldots, n} \rightarrow \set{1, \ldots, n}$ to be a bijective mapping such
    that $\sigma(j) = i_j$ for all $j$ with $1 \leq j \leq m$. For a global
    state $q$, denote by $\pi(q)$ the global state obtained by replacing every
    $i$-th local state in $q$ with the $\sigma(i)$-th local state in $q$.
    Similarly, for a joint action $\alpha$, denote by $\pi(\alpha)$ the joint
    action obtained by replacing every $i$-th local action in $\alpha$ with the
    $\sigma(i)$-th local action in $\alpha$. Then, by the definition of the
    global transition function, we have that $\globaltr n(q, \alpha) = q'$  iff
    $\globaltr n(\pi(q), \pi(\alpha)) = \pi(q')$.  Hence,  $(q, \alpha, q') \in
    \globalrel n$ iff $(\pi(q), \pi(\alpha), \pi(q')) \in \globalrel n$.
    Therefore, ($\msys n, q) \models  \varphi[v_1 \mapsto 1, \ldots, v_m
    \mapsto m]$ iff ($\msys n, \pi(q)) \models  \varphi[v_1 \mapsto \sigma(1),
    \ldots, v_m \mapsto \sigma(m)]$. So, by the initial assumption,  it follows
    that ($\msys n, \pi(\globalinit{n})) \models  \varphi[v_1 \mapsto \sigma(1),
    \ldots, v_m \mapsto \sigma(m)]$. As $\pi(\globalinit{n}) = \globalinit{n}$, we
    have that ($\msys n, \globalinit{n}) \models  \varphi[v_1 \mapsto \sigma(1),
\ldots, v_m \mapsto \sigma(m)]$. Since the set of indices $I$ was arbitrary, we
conclude that $\msys n \models \bictlspec$.  \end{proof}


\section{Formal definition of the abstract global transition function} 
We give the full formal definition of the abstract transition function.
\begin{definition}
  The {\em abstract global transition function} $\globaltr{m}_{ab} : \globalstates{m}_{ab}
  \times \globalacts{m}_{ab} \rightarrow \globalstates{m}_{ab}$ of the abstract 
  system $\sys{m}_{ab}$ satisfies $\globaltr{m}_{ab}(q, \alpha) = q'$ iff the
  following hold:
  \begin{itemize}
    \item $\la{e}{\alpha} \in \prot{e}({\ls{e}{q}})$ and $\tr e(\ls{e}{q},
        \la{e}{\alpha}, A) = \ls{e}{q'}$, where
    $A = \set{\la{i}{\alpha} \mid i \in \set{1,\ldots,m}} \cup
\la{zo}{\alpha}$.
    \item For all $i \in \set{1,\ldots,m, \it{zo}}$, we have that
    $\la{i}{\alpha} \in \prot{i}(\ls{i}{q})$, $\tr i (\ls{i}{q}, \la{i}{\alpha},
    A, \la{e}{\alpha}) = \lprv{i}{q'}$, where  $A = \set{\la{i}{\alpha} \mid i \in
    \set{1,\ldots,m}} \cup \la{zo}{\alpha}$, and $\obs
    i((\lprv{i}{q'},\lper{i}{q}),\ls{e}{q'}) = \lper{i}{q'}$.
  \end{itemize}
\end{definition}


\section{Proof of Theorem 2}
We prove Theorem~2 which we restate here for convenience.
\begin{customthm}{2}
$\msys n \leq_b \masys m$ for any $n \geq m+1$ and $b \geq 0$.
\end{customthm}
\begin{proof}
Let $n \geq m+1$. We show that $\msys n
\leq_b \masys m$ for any $b \geq 0$.
 Define $\gamma_n
\colon \globalstates n \rightarrow \aglobalstates m$ to map concrete states in
$\msys n$ to abstract states in $\masys m$ as follows:
\begin{align*}
  \gamma_n(q) =  \langle &\ls{1}{q}, \ldots, \ls{m}{q}, \\
  &\set{\ls{i}{q} \mid i \in \set{m + 1, \ldots, n}}, \\
  &\ls{e}{q}  \rangle.
\end{align*}

For any $b \geq 0$, define $\sim_b = \set{(q, \gamma_n(q)) \mid q \in
    \globalstates n }$. We show that for any $b \geq 0$, $\sim_b$ is a
    $b$-bounded simulation relation between $\msys n$ and $\masys m$. There are
    two cases: $b=0$ and $b >0$. Consider the case where $b=0$. By the
    definitions of initial concrete and abstract global states we have that
    $(\globalinit n, \aglobalinit m) \in \sim_0$. Let $(q, \abstate) \in
    \sim_0$ be arbitrary. As $\gamma_n$ preserves the local states of the
    agents $1, \ldots, m$, we have that $q \in \valuation n(p, i)$ implies that
    $\abstate \in \avaluation m(p, i)$ for any $p \in \atprop$ and $i \in
    \set{1, \ldots, m}$, thus $\sim_0$ is $0$-bounded simulation between $\msys
    n$ and $\masys m$.  

For the case where $b >0$, 
     we
    show that $\sim_{b}$ is a $b$-bounded simulation relation  between
    $\msys n$ and $\masys m$.  By the definitions of initial concrete and
    abstract global states we have that $(\globalinit n, \aglobalinit m) \in
    \sim_{b}$.  Let $(q, \abstate) \in \sim_b$ be arbitrary. As $\sim_{b} =
    \sim_0$, $(q, q') \in \sim_0$, so $\sim_{b}$ satisfies the
    first condition of bounded simulation. To show that
    it satisfies the second condition,  assume that $(q, \alpha, q') \in
    \globalrel n$ for some joint action $\alpha$ and global state $q'$. We need
    to show that there is an abstract joint action $\abaction$ and an abstract
    global state $\abstatep$ such that $(\abstate, \abaction, \abstatep) \in
    \aglobalrel m$ and $(q', \abstatep) \in \sim_{b-1}$.  Define $\delta_n \colon
    \globalacts n \rightarrow \aglobalacts m$ to map joint actions in $\msys n$
    to joint actions in $\masys m$ as follows:
\begin{align*}
  \delta_n(\alpha) =  \langle &\la{1}{\alpha}, \ldots, \la{m}{\alpha}, \\
    &\set{(\ls{i}{q}, \la{i}{\alpha}) \mid i \in \set{m + 1, \ldots, n}}, \\
  &\la{e}{q}  \rangle.
\end{align*}
Let $\abaction = \delta_n(\alpha)$. By the definition of the abstract protocol
function we have that $\abaction \in \prot{ab}(\abstate)$. So $(\abstate,
\abaction, \abstatep) \in \aglobalrel m$, where $\abstatep = \aglobaltr m
(\abstate, \abaction)$. By the definition of the concrete and abstract local
transition functions, and since the set $\set{\la{i}{\alpha} \mid i \in
\set{1,\ldots,n}}$ of actions in $\alpha$ equals the set $\set{\la{i}{q} \mid i
\in \set{1,\ldots,m}} \cup \set{\alpha_t \mid \exists l_t \colon (l_t, \alpha_t)
\in \la{zo}{\abaction}}$ of actions in $\abaction$, we have that $\ls{i}{q'} =
\ls{i}{\abstate}$ for $i \in \set{1,\ldots,m}$, and $\set{\ls{i}{q'} \mid i \in
\set{m+1,\ldots,n}} = \ls{zo}{\abstate'}$.  Therefore $(q', \abstatep) \in
    \sim_{b-1}$ as required.

We have thus proven that for any $b \geq 0$, $\sim_b = \set{(q, \gamma_n(q))
    \mid q \in \globalstates n }$ is a $b$-bounded simulation relation between
    $\msys n$ and $\masys m$. It follows that $\msys b \leq_b \masys n$ for any
    $b \geq 0$.

\end{proof}


\section{Proof of Theorem 3}
We prove Theorem~3 which we restate here for convenience.
\begin{customthm}{3}
There is $n \geq m+1$ such that $\masys m \leq_b \msys n$ for any $b \geq 0$.
\end{customthm}

\begin{proof}

The proof of the Theorem is by induction on~$b$. For any $n \geq m+1$,  define
$\gamma_n \colon \globalstates n \rightarrow \aglobalstates m$ to map concrete
states in $\msys n$ to abstract states in $\masys m$ as follows:
\begin{align*}
  \gamma_n(q) =  \langle &\ls{1}{q}, \ldots, \ls{m}{q}, \\
  &\set{\ls{i}{q} \mid i \in \set{m + 1, \ldots, n}}, \\
  &\ls{e}{q}  \rangle.
\end{align*}

For the base step, let $b = 0$. Set $n = m+1$. Define $\sim_0 \subseteq 
    \aglobalstates m \times \globalstates{m+1}$ by $(q_{ab}, q) \in \sim_0$ if 
$\gamma_{m+1}(q) = \abstate.$
By the definitions of
    initial concrete and abstract global states we have that $(\aglobalinit m,
    \globalinit {m+1}) \in \sim_0$.  
    As $\gamma_{m+1}$ preserves
the local states of the agents $1, \ldots, m$, we additionally have that $\abstate \in
\avaluation m(p, a)$ implies that $q \in \valuation{m+1}(p, a)$ for any $p \in
\atprop$ and $a \in \set{1, \ldots, m}$. Thus,  $\sim_0$ is $0$-bounded
simulation relation between $\masys m$ and $\msys{m+1}$, hence $\masys m \leq_0 \msys{m+1}$.

 For the inductive step, assume that for each $i \in
    \set{1, \ldots, b}$ there is $n_i \geq m + 1$ such that $\masys m \leq_b
    \msys{n_i}$ by means of a relation $\sim_{n_i}$ satisfying
    $\gamma_{n_i}(q)=q_{ab}$ whenever $(q_{ab}, q) \in \sim_{n_i}$.
    %$$\sim_i =
%\set{(\abstate, q) \mid \abstate \in \aglobalstates{m}, q \in
%\globalstates{n}, \gamma_{n}(q) = \abstate}.$$ 
    We show that there is $n' \geq n_b$ such that $\masys m \leq_{b+1}
    \msys{n'}$ by means of a relation $\sim'_{b+1}$ satisfying
$\gamma_{n'}(q)=q_{ab}$ whenever  $(q_{ab}, q) \in
    \sim_{b+1}$.
    

    Let $|\prot t| = \max \set{|\prot t(l_t)| \mid l_t \in \lstates{t}}$ be the
    maximum number of actions enabled by the template protocol at any local
    state. Set $n' = n_b +  (n_b -m) |\prot t|$.  Given a
    global state $q$ (either concrete or abstract) and $i > 0$, let
    $\projection{q}{i}$ denote the projection of $q$ to the first~$i$ agents,
    i.e. $\projection{q}{i} = \tuple{\ls{1}{q}, \ldots, \ls{i}{q}, \ls{e}{q}}$.
    Define the following relations $\sim'_i \subseteq \aglobalstates{m} \times
    \globalstates{n'}$ between the abstract global states in $\masys{m}$ and the
    concrete global states in $\msys{n'}$:

\begin{itemize}

    \item  $\sim'_0 = \set{(\abstate, q) \mid  \gamma_{n'}(q) = \abstate}$.
        %where $q_{ab \rightarrow m}$ and $\projection{q}{m}$
  %is the projection of $q_{ab}$ and $q$ onto the first~$m$ agents, e.g.,
  %$\projection{q}{m} = \tuple{\ls{1}{q}, \ldots, \ls{m}{q}, \ls{e}{q}}$.


  \item for $i \in \set{1, \ldots, b + 1}$, we have that $(\abstate, q) \in
  \sim'_i$ if:
  \begin{itemize}
    \item  $(\abstate,\projection{q}{n_b}) \in \sim_{i-1}$, and 
    \item  for $j \in \set{1,\ldots,n_b - m}$ and $k \in \set{1,\ldots,|\prot t|}$, 
    $\ls{n_b + (j - 1) |\prot t| + k}{q} = \ls{m+j}{q}$; i.e., there are
          additional $|\prot t|$  agents in each of the local state of each of
          the agents in $\set{m+1,\ldots,n_b}$.
  \end{itemize}
\end{itemize}

We show that for every $x \in \set{0, \ldots, b+1}$, $\sim'_x$ is an
$x$-bounded simulation between $\masys m$ and $\msys{n'}$. There are three
cases: $x=0$, $x=1$, and $x \in \set{2,\ldots,b+1}$.
    
\begin{itemize}
  \item Case 1: $x=0$. By the definition of the abstract and concrete initial
      global states, we have that $(\aglobalinit m, \globalinit{n'}) \in
        \sim'_0$. Let $(q_{ab}, q) \in \sim'_0$ be arbitrary. Since
        $\gamma_{n'}(q) = \abstate$, it follows that $\abstate \in \avaluation m(p,
        i)$ implies that $q \in \valuation n (p, i)$ for any $p \in \atprop$
        and $i \in \set{1,\ldots,m}$, thus $\sim'_0$ is a $0$-bounded
        simulation between $\masys m$ and $\msys{n'}$.


  \item Case 2: $x = 1$.   By the definition of the abstract and concrete
      initial global states, we have that $(\aglobalinit m, \globalinit{n'})
        \in \sim'_1$. Let $(q_{ab}, q) \in \sim'_1$ be arbitrary. By the
        definition of $\sim'_1$, it follows that $(\abstate,
        \projection{q}{n_b}) \in \sim_{0}$, thus $\gamma_{n'}(q) = \abstate$,
        hence  $(\abstate, q) \in \sim'_0$, so $\sim'_1$ satisfies the first
        condition of bounded simulation. 


  To show that it satisfies the second condition, assume that $(\abstate,
  \abaction, \abstatep) \in \aglobalrel m$ for some abstract joint action
  $\abaction$ and abstract global state $\abstatep$. We need to show that there
  is a joint action $\alpha$ and a  global state $q'$ such that $(q, \alpha, q')
  \in \globalrel{n'}$ and $(\abstatep, q') \in \sim'_0$.  
Define $\alpha$ as
  follows:

  \begin{itemize}
      \item $\la{i}{\alpha} = \la{i}{\abaction}$, for $i \in \set{1,\ldots,m}$.
      \item  for $i \in \set{m+1,\ldots,n_b}$, consider the set $\set{a_t \mid (\ls{i}{q},\alpha_t) \in \la{zo}{\abaction}}$ of template actions
          that are paired with the template state of one of the agents in $\set{1,\ldots,n_b}$
    in the action of the zero-one agent. Assume an ordering $\alpha_1, \ldots,
    \alpha_j$ of this set of  actions and define
    \begin{itemize}
        \item $\la{i}{\alpha} = \alpha_1$.
        \item  $\la{n_b + (i -1) |\prot t| + k}{\alpha} = \alpha_k$  for $k \in \set{1,\ldots,j}$.
    \item $\la{n_b + (i-1) |\prot t| + k}{\alpha} = \alpha_1$ for $k \in
    \set{j+1, \ldots, |\prot t|}$.
    \end{itemize}
  \end{itemize}

So the additional $|\prot t|$ agents for each of the agents in
        $\set{m+1,\ldots,n_b}$ collectively perform in $\alpha$ all of the template actions
        that are paired with a local state in the action of  the zero-one
        agent. By the definition of $\alpha$, we have that $\la{i}{\alpha} \in \prot
        i(\ls{i}{q})$ for every $i \in \set{1,\ldots,n'}$. So $(q, \alpha, q') \in \globalrel
  {n'}$, where $q' = \globaltr{n'} (q, \alpha)$. By the definition of the
  concrete and abstract local transition functions, and since the set
        $\set{\la{i}{\alpha} \mid i \in \set{1,\ldots,n'}}$ of actions in $\alpha$ equals the set
  $\set{\la{i}{q} \mid i \in \set{1,\ldots,m}} \cup \set{\alpha_t \mid \exists
  l_t \colon (l_t, \alpha_t) \in \la{zo}{\abaction}}$ of actions in $\abaction$,
        we have that $\gamma_{n'}(q'_{ab}) = q'$, therefore
  $(\abstatep, q') \in \sim'_0$, as required. 


  \item Case 3: $x \in \set{2, \ldots, b+1}$.  By the definition of the
      abstract and concrete initial global states, we have that $(\aglobalinit
        m, \globalinit{n'}) \in \sim'_x$. Let $(q_{ab}, q) \in \sim'_x$ be
        arbitrary. By the definition of $\sim'_x$ we have that $(\abstate,
        \projection{q}{n_b}) \in \sim_{x-1}$, thus $\gamma_{n_b}(\abstate) =
        q$, so $\gamma_{n'}(\abstate) = q$, hence $(\abstate, q) \in \sim'_0$,
        and therefore $\sim'_{x}$ satisfies the first condition of bounded
        simulation. 


  To show that it satisfies the second condition, assume that $(\abstate,
  \abaction, \abstatep) \in \aglobalrel m$ for some abstract joint action
  $\abaction$ and abstract global state $\abstatep$. We need to show that there
  is a joint action $\alpha$ and a  global state $q'$ such that $(q, \alpha, q')
  \in \globalrel{n'}$ and $(\abstatep, q') \in \sim'_{x-1}$. Since $(\abstate,
  \projection{q}{n_b}) \in \sim_{x-1}$, we have that there is a joint action
  $\beta$ and a global state $r$ such that $(\projection{q}{n_b}, \beta, r) \in
  \globalrel{n_b}$ and $(\abstate, r) \in \sim_{x-2}$. Define a joint action
  $\alpha \in \globalacts{n'}$ such that
  
  \begin{itemize}
      \item for $i \in \set{1,\ldots,n_b}$, $\la{i}{\alpha} = \la{i}{\beta}$, 
      \item  for $i \in \set{1, \ldots, n_b-m}$ and $j \in \set{1,\ldots,|\prot
          t|}$, $\la{n_b + (i-1)|\prot t| + j}{\alpha} = \la{m+i}{\beta}$.
  \end{itemize}

  By the definition of $\alpha$, we have that $(q, \alpha, q')  \in
  \globalrel{n'}$ for a global state $q'$ that satisfies

  \begin{itemize}
    \item  $(\abstate',\projection{q'}{n_b}) \in \sim_{x-1}$, and 
    \item  for $j \in \set{1, \ldots, n_b-m}$ and $k \in \set{1,\ldots,|\prot t|}$, 
        $\ls{n_b + (j-1)|\prot t| + k}{q'} = \ls{m+j}{q'}$.
  \end{itemize}

  It follows that $(q'_{ab}, q') \in \sim'_{x-1}$. Consequently, $\sim'_{x}$
  is an $x$-bounded simulation between $\masys m$ and $\msys{n'}$.

\end{itemize}

We have thus proven that for every $x \in \set{0, \ldots, b+1}$, $\sim'_x$ is an
$x$-bounded simulation between $\masys m$ and $\msys{n'}$.  It follows  that
$\masys m \leq_{b+1} \msys{n'}$.

\end{proof}


\section*{Proof of Theorem 4}
We prove Theorem~4 which we restate here for convenience.
\begin{customthm}{4}
For every $n' > n$ we have that $\msys n \leq_b \msys{n'}$ for any $b \geq 0$.
\end{customthm}
\begin{proof}
Let $n' > n$. For any $b \geq 0$, define $\sim_b \subseteq \globalstates n
    \times \globalstates{n'}$ by $(q,q') \in \sim_b$ if
\begin{itemize}
    \item $\ls{1}{q} = \ls{1}{q'}, \ldots, \ls{n}{q} = \ls{n}{q'}$,
    \item $\ls{n+1}{q'}=\ls{1}{q'}, \ldots, \ls{n'}{q'}=\ls{1}{q'}$.
\end{itemize}
We show that $\sim_b$ is a $b$-bounded simulation relation between $\msys n$
    and $\msys{n'}$. By the definition of the initial concrete global states we
    have that $(\globalinit n, \globalinit{n'}) \in \sim_b$. Let $(q, q') \in
    \sim_b$ be arbitrary. For the case of $b=0$, by the definition of the concrete labelling function, 
    it follows that $q \in \valuation n(p, i)$ implies that $q' \in \valuation
    {n'}(p, i)$ for any $p \in \atprop$, $i \in \set{1,\ldots,n}$, thus $\sim_0$ is a $0$-bounded simulation between $\msys n$ and $\msys{n'}$. 

    For the case of $b > 0$, since $\sim_b = \sim_0$, $\sim_b$ satisfies the first condition of bounded simulation. To show that it satisfies the second condition, assume that $(q, \alpha, q^1) \in \globalrel n$ for some joint action
    $\alpha \in \globalacts n$ and global state $q^1 \in \globalstates n$. We
    need to show that there is a joint action $\alpha' \in \globalacts{n'}$ and
    global state $q'^1 \in \globalstates{n'}$ such that $(q', \alpha', q'^1)
    \in \globalrel{n'}$ and  $(q^1, q'^1) \in \sim_{b-1}$.  Define $\alpha'$ by 
    \begin{itemize}
        \item $\la{1}{\alpha'} = \la{1}{\alpha}, \ldots \la{n}{\alpha'}=\la{n}{\alpha}$,
        \item $\la{n+1}{\alpha'} = \la{1}{\alpha}, \ldots, \la{n'}{\alpha'} = \la{1}{\alpha}$.
    \end{itemize}
    By the definition of the conrete transition function we have that $(q',
    \alpha', q'^1) \in \globalrel{n'}$, where $\ls{1}{q'^1} = \ls{1}{q^1},
    \ldots, \ls{n}{q'^1}=\ls{n}{q^1},
    \ls{n+1}{q'^1}=\ls{1}{q'^1},\ldots,\ls{n'}{q'^1}=\ls{1}{q'^1}$. Hence $(q',
    \alpha', q'^1) \in \sim_{b-1}$, and therefore $\sim_b$ is a $b$-bounded simulation relation between $\msys n$ and $\msys{n'}$. 

We have thus proven that for every $b > 0$, there  is a $b$-bounded simulation
    relation between $\msys n$ and $\msys{n'}$. It follows that  $\msys n
    \leq_b \msys{n'}$ for any $b \geq 0$.
\end{proof}


\section{Training of a neural observation function for the Guarding Game} 

We provide more details regarding the training of the neural observation
function for the guarding game.

We used deep Q-learning, a type of reinforcement learning (RL) algorithm.
During the training, the game was played by 4 agents (in the main paper
mistakenly specified as 3), and the game parameters were set as $M_h = 4$,
$G_r = -2$, $R_r = 1$ and $U_r = -3$.
% The rewards were assigned to reflect the tension between individual and
% collective interests.
%
At the beginning, the agents' health was $M_h$.
%
At each turn of the game, the agents that were alive could decide whether they
volunteer to guard or not. When there were volunteers, a seperate arbitration
function decided randomly how many of the volunteers undertook the guarding
action (the other volunteers took the resting action). Each agent had a 50\%
chance of being selected if it volunteered, but it was impossible for no agent
to be selected as long as there were volunteers.

The reward function for each agent consisted of the sum of
\begin{inparaenum}[\it (i)]
\item  a social reward,
\item an individual reward,
\item desire to personally live, and
\item fear of collective failure.
\end{inparaenum}
The social reward for an agent was expressed as the average change in every
other agent's health, minus the fraction of other agents who are dead at the
end of the turn, divided by two (to provide an arithmetic average of the two
values). This was scaled by the agent's normalised health at the end of the
turn. The individual reward was expressed as the change in the agent's health.
The desire to personally live was expressed as plus one point if the agent is
alive at the end of the turn and minus one if it is not, and the fear of
collective failure as minus two if all agents are dead at the end of the turn.



All agents shared the same neural network, creating a system in which the
agents were learning to play against exact copies of themselves.
%
We used an experience replay buffer and a separate, periodically updated target
network for providing estimations.
%
At every turn, all of the experiences (consisting of a state transition and an
associated reward) of the agents in that round were appended to the replay
buffer, which was implemented as a deque. 
%
Each time the policy network was trained, a random minibatch of experiences
(tuples consisting of a state transition and associated reward) was selected
from the replay buffer as training data.
%
The training then proceeded as per \cite{Mnih+15}, and the use of a target
network to stabilise the Q-value estimates during training is described in
the original paper \cite{HaaseltGS16}.
%
The reinforcement learning procedure ran for 1024 episodes of a maximum length
of 16 turns each, if the agents did not all expire. There were 4 agents in the
training simulation. The target network was updated every 32 turns, the replay
buffer held 4096 experiences and was sampled in minibatches of size 16.

The network contained two hidden layers each four units wide and using ReLU
activation functions. The output layer used linear activation and had two units
corresponding to the two actions (rest and guard) available while the agent had
not expired. We used RMSProp gradient descent in minibatches corresponding to
those described in the process of experience replay; the batch size was set to
the number of experiences in each replay sample. The loss function used was
Huber loss, and the actions taken during training utilised an epsilon-greedy
exploration schedule.
% This is as per the original deep Q-learning paper \cite{Mnih+15} and the
% follow up proposing double deep Q-learning, \cite{HaaseltGS16}.
%
For RMSProp, a learning rate of 0.0025 was used, with the default $\rho$ of 0.9
and momentum of 0. For epsilon-greedy exploration, the starting epsilon was 1.0
and it decayed by a factor of 0.999 every turn to a minimum of 0.01. We built
and trained the network using the TensorFlow Keras library.




%% The file named.bst is a bibliography style file for BibTeX 0.99c
\bibliographystyle{named}
\bibliography{../bib}

\end{document}

