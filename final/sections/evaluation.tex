
In this section we present an evaluation of the parameterised verification
procedures described in Section~\ref{sec:verification} on the guarding game
presented in Example~\ref{ex:agent-template}.


%%%%
The guarding game is an instance of a social dilemma game, a class of
MAS characterised by tension between individual
and collective rationality \cite{VanlangeJPV13}. The  game simulates
the fundamental forces of a \emph{collective risk dilemma (CRD)}, a type of
social dilemma where a guaranteed ``tragedy of the commons''
\cite{Hardin68} is avoided by personal sacrifice by a population of
agents, or brought on by free riding if all the agents %favour a strategy that
neglect the collective interests % of that personal sacrifice
\cite{SantosP11}.
%
Namely, guarding can be considered equivalent to cooperation (acting in
collective interest), and resting to defection (acting in selfish
interest).


We train a neural observation function using deep Q-learning, a type of
reinforcement learning (RL) algorithm. During the training, the game was played
by 4 agents, and the parameters were set as $M_h = 4$, $G_r = -2$, $R_r = 1$
and $U_r = -3$. The rewards were assigned to reflect the tension between
individual and collective interests. All agents shared the same neural network,
and thus
% creating a system in which the agents
were learning to play against exact copies of themselves.
%
%\nb{E: commented out experience replay buffer and target network}
%We used an experience replay buffer and a separate target network which was
% updated only periodically to stabilise overestimation of the Q-values
% \cite{Mnih+15,HaaseltGS16}.
% More details about the training can be found in the Appendix.
%
The produced neural network has two hidden layers of four ReLU activated
neurons, takes as input a single neuron, representing the normalised health
points of the agent, and outputs the estimated Q-values of the two actions
`rest' and `guard'.
%%%%


\begin{table}
\centering
% \begin{tabular}{c@{\qquad}rrrrr}
%   \toprule
%   k &\multicolumn{1}{c}{$n=2$} & \multicolumn{1}{c}{$n=3$} & \multicolumn{1}{c}{$n=4$} & \multicolumn{1}{c}{$n=5$} & \multicolumn{1}{c}{$n=6$}\\\midrule
  
%   2 &              0.09s &     0.13s &   0.53s &    1.15s &              5.09s\\
%   3 & \graycell    1.46s &     0.30s &   1.19s &    3.41s &             17.58s\\
%   4 & \graycell    5.49s &     0.52s &   2.31s &   17.74s & \multicolumn{1}{c}{--}\\
%   5 & \graycell   61.47s &   133.28s &   4.28s &   95.83s & \multicolumn{1}{c}{--}\\
%   \bottomrule
% \end{tabular}
\begin{tabular}{c@{\quad}rrrr}
  \toprule
  &
    \begin{tabular}{@{}c@{}}
      $k=2$\\[-0.5mm] ($n=2$)
    \end{tabular}
  &
    \begin{tabular}{@{}c@{}}
      $k=3$\\[-0.5mm] ($n=3$)
    \end{tabular}
  &
    \begin{tabular}{@{}c@{}}
      $k=4$\\[-0.5mm] ($n=3$)
    \end{tabular}
  &
    \begin{tabular}{@{}c@{}}
      $k=5$\\[-0.5mm] ($n=3$)
    \end{tabular}
  \\\midrule
  
  $i=2$ &  0.09s & \graycell  1.46s & \graycell  5.49s & \graycell  61.47s \\
  $i=3$ &  0.13s &            0.30s &            0.52s &           133.28s \\[1mm]
  $i=4$ &  0.53s &            1.19s &            2.31s &             4.28s \\
  $i=5$ &  1.15s &            3.41s &           17.74s &            95.83s \\
  $i=6$ &  5.09s &           17.58s & \multicolumn{1}{c}{--} & \multicolumn{1}{c}{--} \\
  \bottomrule
\end{tabular}
\caption{ Verification times for $\msys{i} \models
    \varphi^k_E[2]$ for various $k$ and $i$. For
    each $k$, we indicate the value of $n$ from
    Corollary~\ref{cor:existential}. Grey cells indicate
    when the property was not satisfied.  Dashes indicate a
    1 hour timeout.  }
  \label{tab:results-existential}
\end{table}

Given the learned neural network, we implemented a template
agent and a zero-one agent for the guarding game. We then
used Corollaries~\ref{cor:universal}
and~\ref{cor:existential} to verify whether it is possible
for a colony of agents to survive after a number of time
steps.  Specifically, recall from Example~\ref{ex:pnis} that
proposition~$\mathsf{a}$ labels all states with positive
health (``alive'') and proposition $\mathsf{d}$ labels all
other states with no health (``dead'').
We considered two
specifications (for $v_i \in\atvar$):
\begin{enumerate}
    \item The existential property $ \forall_{v_1,
        v_2}\varphi^k_E$, where $\varphi^k_E =   EX^k (
        (\mathsf{a},v_1) \land (\mathsf{a}, v_2))$. The
        property expresses that there is an evolution where
        at least~$2$ agents are alive after $k$ time steps.
\item The universal property $\forall_{v_1,\ldots,v_m}
\varphi^k_A$, where $\varphi^k_A  = AX^k
        \bigwedge_{i=1}^m(\mathsf{a},v_i)$. The property
        expresses that in every possible evolution at least
        $m$ agents are alive after $k$ time steps, for $m
        \in \{2,3\}$.
\end{enumerate}
%\begin{inparaenum}[\it (i)]
%\item the existential property
    %$\varphi^k_E =  \forall_{v_1, v_2} EX^k \left( (\mathsf{a},v_2)
    %\land (\mathsf{a}, v_2)\right)$, 
    %%$\varphi^k_E =  \forall_{v_1, v_2} EX^k \bigwedge_{i=1}^m(\mathsf{a},v_i)$ 
  %expressing that there is an evolution where at least~$2$ agents are alive after
  %$k$ time steps,
%\item the universal property $\varphi^k_A = AX^k \bigwedge_{i=1}^m(\mathsf{a},v_i)$
  %expressing that in every possible evolution at least $m$ agents are alive
  %after $k$ time steps.
%\end{inparaenum}

We used the \venmas toolkit~\cite{Akintunde+20b} for
checking the conrete and abstract systems prescribed by
Corollaries~\ref{cor:universal}
and~\ref{cor:existential}.
The experiments %pertaining to the verification of these properties
were performed on a standard PC running Ubuntu 22.04 with 16GB RAM and
processor Intel(R) Core i5-4460. We relied on Gurobi
v10.0~\cite{Gurobi+16a} to solve the mixed integer linear programs generated by
\venmas.

Let $\pnis$ be the
\panos{} for the guarding example. For the existential
property, we check whether at least two agents can stay
alive for $k$ time steps ($\pnis\models \varphi^k_E$), and
whether there is a minimal number (an emergence threshold)
of agents that can  guarantee that. We vary the
temporal depth~$k$ from 2 to 5.  
%
To verify whether
$\pnis\models\forall_{v_1,v_2}\varphi^k_E$, we use
Corollary~\ref{cor:existential} and check whether
$\msys{i}\models\varphi^k_E[2]$ for every
$i\in\{2,\dots,n\}$, where $n$ is the minimal $i$ such that
$\msys{i}$ $k$-bounded simulates $\masys{2}$. We have that 
$n=2$ for $k=2$, and $n=3$ for $k\geq 3$.


Table~\ref{tab:results-existential} presents the outcomes of
the verification queries
$\msys{i}\models\varphi^k_E[2]$ for $i \in
\{2, \ldots, 4\}$.
%
For $k=2$, since  $\msys{2}
\models\varphi^2_E[2]$, we conclude that
$\pnis\models\forall_{v_1,v_2}\varphi^2_E$. This of course
additionally implies that $n=2$ is an emergence threshold
for $\forall_{v_1,v_2}\varphi^2_E$. For $k \geq 3$, since
$\msys{2}\not\models\varphi^k_E[2]$, we
conclude that
$\pnis\not\models\forall_{v_1,v_2}\varphi^k_E$. Still, since
$\msys{3}\models\varphi^k_E[2]$, we obtain
that $n=3$ is an emergence threshold for
$\forall_{v_1,v_2}\varphi^k_E$, so there need to be at least
3 agents present in the colony to ensure a temporal
evolution whereby the colony is viable.
The verification results for $i \in \set{4,5,6}$ reported
by the table (which are not used
to reason about parameterised verification) demonstrate the
increasing computational cost of verifying concrete systems
for increasing number of agents, thereby empirically
justifying the need for parameterised verification.


%This is no
%longer the case for at least 3 time steps as the system
%composed of two agents violates the property:
%$\msys{2}\not\models\forall_{v_1,v_2}\varphi^k_E$. On the
%other hand, we obtain that $n=3$ is the emergence threshold
%for $\forall_{v_1,v_2}\varphi^k_E$ when $k\geq 3$, that is,
%there need to be at least 3 agents for the colony to be
%viable.

%

Concerning the universal property, we verified
$\varphi^k_A[m]$ against the abstract model $\masys{m}$ for
the temporal depths $k \in \set{1,\dots,6}$ and  $m \in
\set{2,3}$. The verification times and results can be found
in Table~\ref{tab:results-universal}. We observe that
because of 
the presence of the zero-one agent, the verification times
are longer than the ones observed in
Table~\ref{tab:results-existential}. We additionally notice
that the property in question is not satisfied even after 1 time step,
which is expected given that there exist paths where no agent is
guarding even when there are volunteers.

In summary, our experimental results confirm the
intractability of verification for concrete models as the
number of agents grows, thereby motivating the need for the
parameterised verification methods that we put forward.


% The arbitrary selection of a subset of guards that volunteer is a necessary
% artefact of \panos{}. Therefore, the evaluation of a universal property like
% $\varphi^k_A[m]$ will consider paths where nobody is guarding even when there
% are volunteers. Nevertheless, we are able to reason about existential
% properties in a meaningful way, as the only event which would incur the tragedy
% of the commons is the lack of any volunteer guarding in the first place.
% % The arbitrary selection of a subset of guards that volunteer is a
% % necessary artefact of the interpreted systems, but does not affect the
% % stability of the system negatively and allows us to reason about its
% % existential properties, as the only event which would incur the tragedy of the
% % commons is the lack of any volunteer guards in the first place.

\begin{table}
  \centering
\addtolength{\tabcolsep}{-0.1em}
  \begin{tabular}{@{}crrrrrr@{}}
    \toprule
    % $k$ & \multicolumn{1}{c}{$m = 2$} & \multicolumn{1}{c}{$m = 3$}\\
    % \midrule
    % 1 & \graycell        0.66s      & \graycell        1.75s\\
    % 2 & \graycell        3.60s      & \graycell       16.03s\\
    % 3 & \multicolumn{1}{c}{--}      & \graycell        0.00s\\
    % 4 & \graycell       16.90s      & \multicolumn{1}{c}{--}\\
    % 5 & \graycell       20.59s      & \multicolumn{1}{c}{--}\\
    % 6 & \multicolumn{1}{c}{--}      & \multicolumn{1}{c}{--}\\

    m & \multicolumn{1}{c}{$k = 1$} & \multicolumn{1}{c}{$k = 2$} & \multicolumn{1}{c}{$k = 3$} & \multicolumn{1}{c}{$k = 4$} & \multicolumn{1}{c}{$k = 5$} & \multicolumn{1}{c}{$k = 6$}\\
    \midrule
2 & \graycell    0.66s & \graycell    3.60s & \multicolumn{1}{c}{--} & \graycell   16.90s & \graycell   20.59s & \multicolumn{1}{c}{--}\\
3 & \graycell    1.75s & \graycell   16.03s & \multicolumn{1}{c}{--} & \graycell 42.74s & \graycell 2196.70s & \multicolumn{1}{c}{--} \\
    \bottomrule
  \end{tabular}
  \caption{ Verification times for $\masys{m} \models \varphi^k_A[m]$ for various
    $m$ and $k$.  Grey cells indicate when the property was not satisfied.
    Dashes indicate a 1 hour timeout.  }
  \label{tab:results-universal}
\end{table}



%%% Local Variables:
%%% mode: latex
%%% fill-column: 79
%%% TeX-master: "../main"
%%% End:
