Safety concerns stemming from the increasing development of Multi-agent Systems
(MAS) have been put under mathematical scrutiny
%% E: commented out
% over the past decade
by automated methods that ascertain their correct behaviour. Verification
methods based on SAT and
BDDs~\cite{KacprzakLomuscioPenczek04b,RaimondiLomuscio05c} have resulted in
push-button engines such as Verics, MCK and
MCMAS~\cite{GammieMeyden04a,Kacprzak+07a,LomuscioQuRaimondi15}.  In conjunction
with increasingly sophisticated state-space reduction techniques, such as
predicate abstraction~\cite{lomuscio2015verifying} and partial order
reductions~\cite{jamroga2020towards}, the verifiers have been able to scale to
the analysis of systems with very large state spaces.


While the different methods target the provision of effective solutions to
different types of analyses, e.g., fast search for
counterexamples~\cite{penczek2003verifying} as opposed to fast correctness
proofs~\cite{ball2006abstraction}, and for different classes of MAS, e.g., MAS
defined over infinite-state as opposed to finite-state
variables~\cite{lomuscio2015verifying}, all methods make two fundamental
assumptions. The first is that the MAS under analysis is composed of a known
number of agents specified at design time. The second is that the agents
composing the MAS are specified using traditional programming languages.  The
approaches cannot
%% E
% in principle
therefore be used to verify important classes of MAS, where either the systems
have arbitrarily many participants or the agents are endowed with machine
learning components. The former class of systems includes open systems, where
agents can join and leave the system at runtime, and applications designed
irrespective of the number of participants, such as robot
swarms and scenarios
in the Internet of Things. The latter class comprises forthcoming
neural-symbolic applications such as autonomous vehicles.

More recent methods have addressed the verification of
systems with an unbounded number of
constituents~\cite{KouvarosLomuscio16a,KouvarosLomuscio16c}.
While the different methods focus on different communication
primitives for the agents, most of them rely on abstractions
whereby the unbounded verification problem is reduced to
analysing a finite state-space. The resulting techniques
formed the backbone of various formal reasoners such as
those targeting
fault-tolerance~\cite{KouvarosLomuscioPirovano18} and
data-aware systems~\cite{BelardinelliKouvarosLomuscio17}.

In a different line of work, verification methods for MASs comprising agents
with neural components were put forward~\cite{Akintunde+20b,Akintunde+22}. To
deal with the real-valued operational domain of the neural network models, the
methods recast verification queries for bounded properties into
Mixed-Integer-Linear-Programming.

While these lines of work independently tackle unbounded and neural-symbolic
MAS, none of the underlying methods can be used for analysis of systems that are
both unbounded and neural-symbolic.  In this paper we overcome this limitation.
Specifically, we introduce \emph{Parameterised Neural Interpreted Systems (PNIS)}, a
formal model
%% E:
% based on neural interpreted systems 
for modelling unbounded neural-symbolic MAS.  We develop an abstraction
methodology for PNIS whereby we derive \textcolor{red}{novel} sound and complete procedures for the
verification and emergence identification problems with respect to bounded
universal and existential CTL formulae.  We use these procedures to 
analyse a social dilemma scenario.
% utilise an implementation of these procedures to analyse a social dilemma
% scenario.

% to analyse the temporal and emergent properties of 
% neural-symbolic agents playing a social dilemma game,
% We discuss the significance of solutions to the verification and emergence
% identification problems in the context of the subset of social dilemmas to
% which our methodology is applicable.

% We additionally present an example of these procedures applied to
% neural-symbolic agents playing the guarding game, a social dilemma game we describe.
% We discuss the significance of solutions to the verification and emergence
% identification problems in the context of the subset of social dilemmas to
% which our methodology is applicable.

% \end{itemize}

The rest of the paper is organised as follows. After discussing related work
below, we present PNIS in Section~\ref{sec:pnis}, followed by the development of
the verification and emergence identification procedures in
Section~\ref{sec:verification}, which we evaluate in Section~\ref{sec:eval}  on
a social dilemma scenario. We conclude in Section~\ref{sec:conclusions}.


{\bf Related Work.}  The contribution is related to the two
lines of work discussed above, namely parameterised
verification and verification for neural-symbolic MAS.
Previous models in parameterised verification targeted
either arbitrarily many agents operating in fixed
environments~\cite{KouvarosLomuscio16a,FelliGianolaMontali20}
or a fixed number of agents living in environments of
arbitrary size~\cite{AminofMuranoRubin16}. Systems
comprising  homogeneous agents were also analysed within the
framework of Alternating-time Temporal Logic but in a fixed,
non-parameterised setting~\cite{PedersenDyrkolbotn13}. All
of these works do not include neural components, which
require the formulation of novel abstractions. Also
differently from previous work, which is typically concerned
with the analysis of universal properties, the neural
components here considered restrict the analysis to bounded
temporal formulae, thereby enabling the derivation of sound
and complete verification methods for both universal and
existential (bounded) properties.

Existing verification methods for MAS with neural
components~\cite{Akintunde+20b,Akintunde+22} take as input systems with a known
number of agents. The main theoretical finding of this work is that
verification in the unbounded\nb{E: we use bounded/ unbounded in two different
  ways. One to refer to the temporal bound. The other to refer to the bounded
  number of agents. Maybe we should rephrase it to avoid confusion} case can be
reduced to the verification of a small number of bounded models.% from the
% cited works.



% hese include
% SAT-based and BDD-based verification methods (Kacprzak,
% Lomuscio, and Penczek 2004; Raimondi and Lomuscio
% 2005). Current model checkers, such as Verics, MCK and
% MCMAS (Kacprzak et al. 2008; Gammie and van der Mey-
% den 2004; Lomuscio, Qu, and Raimondi 2015), can effi-
% ciently verify large state-spaces.

% Recent advances in  Artificial Intelligence (AI)  enabled the automation of
% challenging tasks, such as computer vision, that have been traditionally
% difficult to tackle for decades. This accelerated the incorporation of  AI
% components in  diverse applications, including ones situated within domains, such as
% healthcare and transportation, where the impact to society can be significant.
% While however AI has the potential of revolutionising society,  its inherent
% fragility and opacity   hinders its adoption in safety-critical applications.
% The associated risks are compounded in an increasingly inter-connected
%Still, even though there is an increasing consensus in AI being beneficial for
%society,  its inherent fragility and opacity hinders its adoption in
%safety-critical applications. The associated risks are compounded in an
%increasingly inter-connected
% socio-techno-economic
% world, where systems of
% multiple interacting intelligent agents, or multi-agent systems (MAS),
% constitute a paradigm shift from object-oriented to interaction-oriented design
% standards. 

% In response to these concerns the area of formal verification of AI has grown
% rapidly over the past few years to provide methods to automatically verify that
% AI systems robustly behave as intended.
% One of the key techniques that has emerged in the area is that of {\em model
% checking}~\cite{Clarke+99a}. Model checking provides automated solutions to 
% the problem of establishing whether a model $M_S$ representing a system $S$ satisfies a
% logical formula $\phi_P$ encoding a specification $P$. In the case of MAS,
% the formula $\varphi$ does not simply express temporal properties of systems, as
% in reactive systems, but it may also be accounting for high-level attitudes of
% agency, such as knowledge and strategies, which can be described in
% temporal-epistemic logic~\cite{Fagin+95b} and alternating-time
% logic~\cite{Alur+98a}.

% Whilst methods such as binary binary decision diagrams~\cite{GammieMeyden04a}
% and bounded model checking~\cite{PenczekLomuscio03b}   enabled the  model
% checking  of systems of large state spaces, a main drawback of the approach
% remains the state-space explosion problem, whereby the state-space grows
% exponentially in the number of variables encoding the agents.

%growth of the state space in the number of
%variables encoding the agents.

% Notwithstanding that in practice this limits model checking to the verification
% of systems with only few constituents, the analysis of  systems with arbitrarily many
% participants, such as robot swarms  and  applications in the Internet of Things, 
% raises a principal barrier to its application.  Indeed, 
% verifying systems of this kind, henceforth {\em unbounded multi-agent systems}
% (UMAS), requires checking whether any system for any number of agents satisfies
% the specification in question. This  renders model checking intractable when 
% enumerating and analysing all individual systems. 


% The formal verification problem is concerned withV
% $S$ satisfies a safety property $P$. {\em Model checking}, a key method for the
% formal verification of reactive systems, has also been used in the past fifteen
% years to provide automated solutions to this problem~\cite{Clarke+99a}. In
% model checking, the system is represented as a model $M_S$, the specification
% is encoded as formula $\varphi$ and it is then checked whether $M_S$ satisfies
% $\varphi$. 

% In these cases, one could encode a system with a given number of agents and
% verify that a specification holds. However, additional agents may possibly
% interfere with the system in unpredictable ways resulting in the specifications
% being violated. Therefore, to fully verify the system, the process would have to
% be repeated for any possible number of components

% Another key limitation of model checking is the requirement that the systems are
% given in traditional, agent-based programming languages, thereby not accounting for  
%  agents endowed with neural network components.
% Systems of this kind, henceforth {\em neuro-symbolic multi-agent systems}
% (NMAS), constitute important forthcoming applications, such as autonomous
% vehicles, where the neural components are responsible for automating complex tasks such
% as perception and control. 

% Even though neural networks exhibit remarkable performance on these tasks,
% their fragility to adversarial attacks~\cite{Szegedy+14} and
% their lack of interpretability~\cite{VelizKim17} raise additional concerns
% regarding the overall system safety, thereby strengthening the need for the
% principled analysis of NMAS before deployment.

% This paper gives an overview of the methods that we developed within the
% Verification of Autonomous Systems Research
% Group, Imperial College London, 
% %~\footnote{\url{https://vas.doc.ic.ac.uk}} 
% towards the formal verification
% of UMAS and NMAS.  Our pioneering work in the verification of UMAS, discussed in
% Section~2, overcomes the model checking barrier with the development of methods
% that enable the derivation of the number of agents that is sufficient to
% consider when evaluating a specification. Our studies in the analysis of NMAS,
% outlined in Section~3, include efficient methods for the verification of neural
% networks and mixed-integer linear programming (MILP) formulations for checking
% system-level specifications.  The paper concludes in Section~4 with directions
% for future work.

%%% Local Variables:
%%% mode: latex
%%% fill-column: 79
%%% TeX-master: "../main"
%%% End:
