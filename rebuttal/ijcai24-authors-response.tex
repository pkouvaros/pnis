% IJCAI 2024 Author's response

% Template file with author's response

\documentclass{article}
\pdfpagewidth=8.5in
\pdfpageheight=11in
\usepackage{ijcai24-authors-response}


\usepackage{times}
\usepackage{soul}
\usepackage{url}
\usepackage[hidelinks]{hyperref}
\usepackage[utf8]{inputenc}
\usepackage[small]{caption}
\usepackage{graphicx}
\usepackage{amsmath}
\usepackage{amsthm}
\usepackage{booktabs}
\usepackage{algorithm}
\usepackage{algorithmic}
\usepackage{lipsum}
\usepackage{amsmath,amssymb,mathtools}
\urlstyle{same}

\newtheorem{example}{Example}
\newtheorem{theorem}{Theorem}

\begin{document}

% \noindent\fbox{
% 	\parbox{\linewidth}{
% 		{\bf IMPORTANT NOTE (YOU CAN DELETE IT AFTER READING)}

% 		The response to the reviews {\bf cannot} include links to external resources. Your paper may be summarily rejected if you include such links in your response.

% 		Furthermore, it {\bf must be one page-only} and use this template without altering fonts, font sizes, or margins.

% 		In our experience, it is more efficient to focus on explicit questions asked in the reviews (especially those listed in the box ``Questions for the authors response'') than to enter into an argument with the reviewers about the significance/novelty of your results. However, feel free to use this space as you see fit, subject to the constraints above.
% 	}
% }

\section*{Rebuttal}
We thank the reviewers for the insightful comments and for their time in
reviewing a highly technical paper.


\subsection*{Reviewers 1 and 2}
All restrictions to our specification language are concomitant to previous
results pertaining to either undecidability or incompleteness of verification.
In particular, the restriction to bounded properties follows the undecidability
of verification for the unbounded case and the exclusion of negation follows
the incompleteness of verification for concrete models (Akintunde et al.,
2020a). We will give more emphasis to these limitations, should the paper be
accepted.

\subsection*{Reviewer 1} 

\textbf{Q1}.  Theoretically the observation function is any deterministic
function, including concrete neural network models. In practice, verification
for concrete models is only complete for piecewise-linear functions. These
include ReLU-based neural networks. We refer the reviewer to (Akintunde et al.,
2020a) for a detailed discussion.

\noindent
\textbf{Q2}. {\em Negation.} All of our results still hold if
negation would be allowed in our syntax. However, verification methods for
concrete models, including VENMAS used in the present contribution, do not
support negation without hindering the completeness of verification (Akintunde
et al., 2020a). 

{\em Quantification.} The universal quantification of the variables indexing
the atomic propositions is standard in parameterised verification (R. Bloem, et
al., Decidability of Parameterized Verification)  and aims at the expression of
collective properties for the systems. Still, following the symmetry reduction
lemma (Lemma 1), a universally quantified formula holds iff  an existentially
quantified formula holds, so no expressitivity is lost with our syntax.
We will clarify the above points, should the
paper be accepted.

\noindent
\textbf{Q3}. The  work cited by the reviewer considers systems with a known
number of agents operating within an environment of arbitrary size. Differently
from the cited work our systems are composed of an arbitrary number of agents
operating in an environment of a known size. We add a discussion on the cited
work,  should the paper be accepted.


\noindent
\textbf{Work by Pedersen and Dyrkolboth.} We thank the reviewer for bringing to
our attention this work. Similarly to our contribution the work considers
systems with homogeneous agents. But, differently from this submission, it is
concerned with purely symbolic and non-parameterised systems. We will discuss
the relevance of the work to ours, should the paper be accepted.


\noindent
\textbf{Q4 and Q5}
In Corollaries 1 and 2 we show how parameterised
verification can be reduced to non-parameterised verification. Each
single cell in Tables 1 and 2 reports the result of verifying a
concrete system. From these we can draw conclusions regarding
parameterised verification.

In particular, in Table 1, we have that $m=2$ for all experiments $m=2$. For
$k=2$, $n$ from Corollary 2 equals 2. So, as per Corollary 2, from
$\mathcal{S}^{(n)} \models \varphi^2_E[2]$ we conclude that $\mathcal{S}$
satisfies $\forall_{v_1,v_2} \varphi^2_E$ (at least two agents survive after 2
time steps). As for 3 time steps, $k=3$, $n$ from Corollary 2 equals 3. So we
obtain that $\mathcal{S}$ does not satisfy $\forall_{v_1,v_2} \varphi^3_E$ (it
is not the case that at least two agents survive after 3 time steps) because of
the concrete system with 2 agents not satisfying the formula. On the other hand
we obtain that $n$ is an \emph{emergence threshold}. Similarly for $k>3$.

We note that in Table 1 the results for $n\geq 4$ are not used to reason about
parameterised verification. They are instead  included to demonstrate the
increasing computational cost of verifying concrete systems for increasing
number of agents, thereby justifying the need for parameterised verification. 

From Table 2, we conclude that
$\mathcal{S}\not\models\forall_{v_1,\dots,v_m}\varphi^k_A$ for any of
$k=1,\dots,5$ and $m=2,3$ and that there is no emergence threshold.

We will make these explanations explicit in the final
version of the paper should it be accepted.

\textbf{Other minor comments.} Noted with thanks.


\subsection*{Reviewer 2}

\textbf{Q1}. Yes, thanks for spotting this!

\noindent
\textbf{Q2}.  
Concerning universal properties, the size of the abstract model from
Corollary~1  is exponential in the size of the agent template, and its
verification is in {\sc coNExpTime} and {\sc PSpace-hard} [Akintunde et al.,
2020a]. The analysis of the complexity of verifying existential properties
(Corollary~2) is more involved and beyond the scope of the paper. We agree with
the reviewer  that this is an important direction for future work.  
We will add a short discussion on complexity, should the
paper be accepted.

\noindent
\textbf{Q3}. Bounded until $A(\varphi U^k \psi)$ can be inductively abbreviated
using our syntax as follows:
 \begin{align*}
  A(\varphi U^1 \psi) &\triangleq \psi \lor (\varphi \land AX^1 \psi) \\
  A(\varphi U^k \psi) &\triangleq \psi \lor (\varphi \land AX^1 (A (\varphi U^{k-1} \psi)).
 \end{align*}
The dual until, prefixed by the existential operator, can be defined
analogously. Unbounded until cannot be supported in our framework since
verification for unbounded until is undecidable (Akintunde et al., 2020a). We
will describe the above abbreviations in the final version of  the paper,
should it be accepted.

\textbf{Experimental results.} We refer the reviewer to the answers to the
questions Q4 and Q5 of Reviewer 1 for a short discussion. The results show the
intractability of verification for concrete models as the number of agents
grows, thereby motivating the need for the parameterised verification methods
that we put forward. As explained above our methods reduce verification for
systems of any size to verifying only a small number of systems with a small
number of constituents.


\subsection*{Reviewer 3}

\textbf{Novelty:} the paper is the first to enable the formal verification of
multi-agent systems with an unbounded number of neural-symbolic agents. This
required the construction of a novel zero-one abstraction methodology for
parameterised neural interpreted systems.

\noindent
\textbf{Significance:} The contribution reduces an  infinite-state verification
problem to a finite-state one for an important class of multi-agent systems.





\end{document}

