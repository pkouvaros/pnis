% IJCAI 2024 Author's response

% Template file with author's response

\documentclass{article}
\pdfpagewidth=8.5in
\pdfpageheight=11in
\usepackage{ijcai24-authors-response}


\usepackage{times}
\usepackage{soul}
\usepackage{url}
\usepackage[hidelinks]{hyperref}
\usepackage[utf8]{inputenc}
\usepackage[small]{caption}
\usepackage{graphicx}
\usepackage{amsmath}
\usepackage{amsthm}
\usepackage{booktabs}
\usepackage{algorithm}
\usepackage{algorithmic}
\usepackage{lipsum}
\urlstyle{same}

\newtheorem{example}{Example}
\newtheorem{theorem}{Theorem}

\begin{document}

\noindent\fbox{
	\parbox{\linewidth}{
		{\bf IMPORTANT NOTE (YOU CAN DELETE IT AFTER READING)}

		The response to the reviews {\bf cannot} include links to external resources. Your paper may be summarily rejected if you include such links in your response.

		Furthermore, it {\bf must be one page-only} and use this template without altering fonts, font sizes, or margins.

		In our experience, it is more efficient to focus on explicit questions asked in the reviews (especially those listed in the box ``Questions for the authors response'') than to enter into an argument with the reviewers about the significance/novelty of your results. However, feel free to use this space as you see fit, subject to the constraints above.
	}
}

\section*{Rebuttal}
\subsection*{Reviewer 1}
\textbf{Q4 and Q5} In Corollaries 1 and 2 we show how parameterised
verification can be reduced to non-parameterised verification. Each
single cell in Tables 1 and 2 reports the result of verifying a
concrete system. From these we draw conclusions regarding
parameterised verification.

In Table 1, in all experiments $m=2$.
%
For $k=2$, $n$ from Corollary 2 equals 2. So according to Corollary 2,
from $\mathcal{S}^{(n)} \models \varphi^2_E[2]$ we conclude that
$\mathcal{S}$ satisfies $\forall_{v_1,v_2} \varphi^2_E$ (at least two
agents survive after 2 time steps).

As for 3 time steps, $k=3$, $n$ from Corollary 2 equals 3. So we
obtain that $\mathcal{S}$ does not satisfy
$\forall_{v_1,v_2} \varphi^3_E$ (it is not the case that at least two
agents survive after 3 time steps) because of the concrete system with
2 agents. On the other hand we obtain that $n$ is an \emph{emergence
  threshold}. Similarly for $k>3$.

% Strictly speaking, in Table 1 the results for $n\geq 4$ are not used
% to conclude about parameterised verification and are included to give
% an idea about computational efficiency.

From Table 2, we conclude that
$\mathcal{S}\not\models\forall_{v_1,\dots,v_m}\varphi^k_A$ for any of
$k=1,\dots,5$ and $m=2,3$ and that there is no emergence threshold. We
will make it more clear should the paper be accepted.



\end{document}

