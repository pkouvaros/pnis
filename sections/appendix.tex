\subsection*{Proof of Lemma 1} 
We prove the symmetry reduction Lemma which we restate here for convenience:
\begin{customlemma}{1}
$\pnis \models \bictlspec$ iff $\pnis \models \varphi[v_1 \mapsto 1, \ldots, v_m
\mapsto m]$.
\end{customlemma}
\begin{proof}
Let $n \in \mathbb N$ such that $n \geq m$ be arbitrary. We show that $\msys n
\models \bictlspec$ iff $\msys n \models \varphi[v_1 \mapsto 1, \ldots, v_m
\mapsto m]$. The Lemma then follows.

$\boldsymbol{(\Longleftarrow)}$ For the right to left direction assume that $\msys n \models
\bictlspec$. Then, by the definition of satisfaction of $\bictl$
(Definition~\ref{def:sat}), we have that for all $h: \set{v_1, \ldots, v_m}
\rightarrow \set{1,\ldots,n}$,  it holds that $\msys n \models \varphi[v_1
\mapsto h(v_1), \ldots, v_m \mapsto h(v_m)]$.  Hence, $\msys n \models
\varphi[v_1 \mapsto 1, \ldots, v_m \mapsto m]$.


$\boldsymbol{(\Longrightarrow)}$ For the left to right direction assume that $\msys n \models
\varphi[v_1 \mapsto 1, \ldots, v_m \mapsto m]$. Let $I = \set{i_1, \ldots, i_m}
\in \set{1, \ldots, n}^m$ be an arbitrary set of $m$ distinct integers within
$\set{1,\ldots,n}$.  Consider $\sigma : \set{1, \ldots, n} \rightarrow \set{1,
\ldots, n}$ be a bijective mapping such that $\sigma(j) = i_j$ for all $j$ with
$1 \leq j \leq m$. For a global state $q$, denote by $\pi(q)$ the global state
obtained by replacing every $i$-th local state in $q$ with the $\sigma(i)$-th
local state in $q$. Similarly, for a joint action $\alpha$, denote by
$\pi(\alpha)$ the joint action obtained by replacing every $i$-th local action
in $\alpha$ with the $\sigma(i)$-th local action in $\alpha$. Then, by the
definition of the global transition function
(Definition~\ref{def:globaltransition}), we have that $\globaltr n(q, \alpha) =
q'$  iff $\globalrel n(\pi(q), \pi(\alpha)) = \pi(q')$.  Hence,  $(q, \alpha,
q') \in \globalrel n$ iff $(\pi(q), \pi(\alpha), \pi(q')) \in \globalrel n$.
Therefore, ($\msys n, q) \models  \varphi[v_1 \mapsto 1, \ldots, v_m \mapsto m]$
iff ($\msys n, \pi(q)) \models  \varphi[v_1 \mapsto \pi(1), \ldots, v_m \mapsto
\pi(m)]$. So, by the initial assumption,  it follows that ($\msys n,
\pi(\globalinit{n})) \models  \varphi[v_1 \mapsto \pi(1), \ldots, v_m \mapsto
\pi(m)]$. As $\pi(\globalinit{n}) = \globalinit{n}$, we have that ($\msys n,
\globalinit{n}) \models  \varphi[v_1 \mapsto \pi(1), \ldots, v_m \mapsto
\pi(m)]$. Since the set of indices $I$ was arbitrary, we conclude that $\msys n
\models \bictlspec$.  \end{proof}


\subsection*{Formal defintion of the abstract global transition function} In the
following we give the full formal definition of the abstract transition function.
\begin{definition}
  \label{def:globaltransition}
  The {\em global transition function} $\globaltr{m}_{zo} : \globalstates{m}_{zo}
  \times \globalacts{m}_{zo} \rightarrow \globalstates{m}_{zo}$ of the zero-one
  system $\sys{m}_{zo}$ satisfies $\globaltr{m}_{zo}(q, \alpha) = q'$ iff the
  following hold:
  \begin{itemize}
    \item $\la{e}{\alpha} \in \prot{e}({\ls{e}{q}})$ and $\tr e(\ls{e}{q},
    \la{e}{q}, A) = q'$, where
    $A = \set{\la{a}{q} \mid a \in \set{1,\ldots,n}} \cup
\la{zo}{q}$.
    \item For all $a \in \set{1,\ldots,m, \it{zo}}$, we have that
    $\la{a}{\alpha} \in \prot{a}(\ls{a}{q})$, $\tr a (\ls{a}{q}, \la{i}{\alpha},
    A, \la{e}{\alpha}) = \lprv{a}{q'}$, where  $A = \set{\la{a}{q} \mid a \in
    \set{1,\ldots,n}} \cup \la{zo}{q}$, and $\obs
    a((\lprv{a}{q'},\lper{a}{q}),\ls{e}{q'}) = \lper{a}{q'}$.
  \end{itemize}
\end{definition}


\subsection*{Proof of Theorem 1}
We prove the $\bictl$ preservation Theorem which we restate here for
convenience:
\begin{customthm}{1}
$\masys m \models \varphi[v_1 \mapsto 1, \ldots, v_m \mapsto m]$ iff 
$\pnis \models  \varphi[v_1 \mapsto 1, \ldots, v_m \mapsto m]$.
\end{customthm}

\begin{proof}
We first prove the left to right direction and then the right to left direction.

$\boldsymbol{(\Longrightarrow)}$ For the left to right direction assume that  $\masys m \models
\varphi[v_1 \mapsto 1, \ldots, v_m \mapsto m]$.  To establish the preservation
of $\bictl$ formulae from the abstract model to the concrete models we show that
$\masys m$ simulates every concrete model $\msys n$ for $n \geq m$. Intuitively
a model simulates another if every behaviour exhibited by the simulated model is
also exhibited by the simulating model.  Since a $\bictl$ formula may only
quantify over all paths, every formula satisfied by the simulating model is also
satisfied by the simulated model.  Formally,  a relation $\sim \subseteq
\globalstates n \times \aglobalstates m$ is a simulation between $\msys n$ and
$\masys m$ if the following conditions hold:
\begin{enumerate}
  \item $(\globalinit n, \aglobalinit m) \in \sim$.
  \item Whenever $(q, \abstate) \in \sim$, then
  \begin{enumerate}
    \item If $q \in \valuation n(p,a)$, for some $p \in \atprop$ and $a \in
    \set{1,\ldots,m}$, then $\abstate \in \avaluation m(p, a)$.
    \item If $(q, \alpha, q') \in \globalrel n$ for a joint action $\alpha$ and
    global state $q'$, then there is an abstract joint action
    $\abaction$ and an abstract global state $\abstatep$ such that
    $(\abstate, \abaction, \abstatep) \in \aglobalrel m$ and
    $(q', \abstatep) \in \sim$. 
    \end{enumerate}
\end{enumerate}
If there is a simulation relation between $\msys n$ and $\masys n$, then we say
that $\masys m$ simulates $\msys n$. If $\masys m$ simulates $\msys n$, then
$\masys n \models \varphi[v_1 \mapsto 1, \ldots, v_m \mapsto m]$ implies that
$\msys n \models \varphi[v_1 \mapsto 1, \ldots, v_m \mapsto
m]$~\cite{ClarkeGrumbergLong94} (the result in the cited work is shown w.r.t. 
arbitrary Kripke structures and unbounded $\ctl$ formulae).

In the following we show that $\masys m$ simulates $\msys n$. Define $\gamma_n
\colon \globalstates n \rightarrow \aglobalstates m$ to map concrete states in
$\msys n$ to abstract states in $\masys m$ as follows:
\begin{align*}
  \gamma_n(q) =  \langle &\ls{1}{q}, \ldots, \ls{m}{q}, \\
  &\set{\ls{i}{q} \mid i \in \set{m + 1, \ldots, n}}, \\
  &\ls{e}{q}  \rangle.
\end{align*}
We show that $\sim = \set{(q, \gamma_n(q)) \mid q \in \globalstates n }$ is a
simulation relation between $\msys n$ and $\masys m$. By the definitions of the
initial global states of abstract and concrete models the relation satisfies the
first condition of the simulation. 

To show that it satisfies the second condition, let $(q, \abstate) \in \sim$ by
arbitrary. As $\gamma_n$ preserves the local states of the agents $1, \ldots,
m$, we have that $q \in \valuation n(p, a)$ iff $\abstate \in \avaluation(p, a)$
for any $p \in \atprop$ and $a \in \set{1, \ldots, m}$, thus the first clause of
the second condition is satisfied. 

For the second clause,  assume that $(q,
\alpha, q') \in \globalrel n$ for some joint action $\alpha$ and global state
$q'$. We need to show that there is an abstract joint action $\abaction$ and
an abstract global state $\abstatep$ such that $(\abstate, \abaction, \abstatep) \in
\aglobalrel m$ and $(q', \abstatep) \in \sim$.  Define $\delta_n \colon
\globalacts n \rightarrow \aglobalacts m$ to map joint actions in $\msys n$ to
joint actions in $\masys m$ as follows:
\begin{align*}
  \delta_n(\alpha) =  \langle &\la{1}{q}, \ldots, \la{m}{q}, \\
  &\set{\la{i}{q} \mid i \in \set{m + 1, \ldots, n}}, \\
  &\la{e}{q}  \rangle.
\end{align*}
Let $\abaction = \delta_n(\alpha)$. By the definition of the abstract protocol
function we have that $\abaction \in \prot{zo}(\abstate)$. So $(\abstate,
\abaction, \abstatep) \in \aglobalrel m$, where $\abstatep = \aglobaltr m
(\abstate, \abaction)$. By the definition of the concrete and abstract local
transition functions, and since the set $\set{\la{i}{\alpha} \mid i \in
\set{1,\ldots,n}}$ of actions in $\alpha$ equals the set $\set{\la{i}{q} \mid i
\in \set{1,\ldots,m}} \cup \set{\alpha_t \mid \exists l_t \colon (l_t, \alpha_t)
\in \la{zo}{\abaction}}$ of actions in $\abaction$, we have that $\ls{i}{q'} =
\ls{i}{\abstate}$ for $i \in \set{1,\ldots,m}$, and $\set{\ls{i}{q'} \mid i \in
\set{m+1,\ldots,n}} = \ls{zo}{\abstate'}$. Therefore $(q', \abstatep) \in \sim$
as required.



$\boldsymbol{(\Longleftarrow)}$  For the right to left direction  assume that
$\pnis \models  \varphi[v_1 \mapsto 1, \ldots, v_m \mapsto m]$. For
contradiction assume that $\masys m \not \models \varphi[v_1 \mapsto 1, \ldots,
v_m \mapsto m]$.  Then, there is a finite path $\abpath$
of evidencing the dissatisfaction of the formula. 

We construct by induction on the length $len(\abpath)$ of $\abpath$ a concrete
path $\rho$ in $\masys n$, $n \geq m$, such that $\gamma_n(\abpath(j)) =
\rho(j)$ for $j \in \set{1,\ldots,len(\abpath)}$.  For the base case of
$len(\abpath)=1$, let $\rho = \globalinit n$ for any $n > m$.  We have that
$\gamma_n{\abpath(1)} = \rho{1}$, thus the
base case is satisfied.

For the inductive case assume that $len(\rho) = x$ and that we have constructed
a concrete path $\rho$ in $\masys n$, $n > m$, of the same length such that
$\gamma_n(\abpath(j)) = \rho(j)$ for $j \in \set{1,\ldots,x}$. We show that a
similar path can be constructed for $len(\abpath) = x+1$.  

Let $\abaction$ be the joint abstract action for which $(\abpath(l), \abaction,
\abpath(x+1)) \in \aglobalrel m$. For any template local state $l_t$ let
$\#(\abaction, l_t)$ denote the number of state-action pairs involving $l_t$ in
$\la{zo}{\abaction}$, i.e., $\#(\abaction, l_t) = \set{\alpha_t \mid (l_t,
\alpha_t) \in \la{zo}{\abaction}}$. Consider $\it{diff}(l_t)$ to be the
difference between $\#(\abaction, l_t)$ and the number of agents from $\set{m+1,
\ldots, n}$ that are in local state $l_t$ in $\rho(x)$, i.e.,  $\it{diff}(l_t) =
\#(\abaction, l_t) - \#\set{i \mid i \in \set{m+1,\ldots,m} \text{ and }
\ls{i}{\rho(x)} = l_t}$. 


Let $n' = \sum_{l_t} \max( \it{diff}(l_t), 0) + n$.  We construct a concrete
path $\rho'$ in $\msys{n'}$ by specifying the sequence of actions of each of the
agents.  The first $n$ agents perform the exact sequence of actions of the
agents in $\masys n$. For each $l_t$ with $\it{diff}(l_t) > 0$, the next
$\it{diff}(l_t)$ agents perform the same sequence of actions as the first agent
that is in state $l_t$ in $\rho(x)$ (by the inductive hypothesis there is at
least one agent in every $l_t$ with $\#(\abaction, l_t) \geq 1$). By
construction the agents $\set{m+1, \ldots, n'}$ in $\rho'(x)$ can collectively
perform the set of actions $\la{zo}{\abpath(x)}$ performed by the zero-one agent
in $\abpath(x)$. This transitions the system to a state $\rho'(x+1)$ such that
$\gamma_n(\abpath(x+1)) = \rho'(x+1)$ as required.

We therefore have that $\msys n'  \not \models \varphi[v_1 \mapsto 1, \ldots,
v_m \mapsto m]$, thus $\pnis \not \models \varphi[v_1 \mapsto 1, \ldots,
v_m \mapsto m]$, concluding the Theorem.

\end{proof}




