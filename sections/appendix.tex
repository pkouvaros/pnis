\subsection*{Proof of Lemma 1} 
We prove the symmetry reduction Lemma which we restate here for convenience:
\begin{customlemma}{1}
$\pnis \models \bictlspec$ iff $\pnis \models \varphi[v_1 \mapsto 1, \ldots, v_m
\mapsto m]$.
\end{customlemma}
\begin{proof}
Let $n \in \mathbb N$ such that $n \geq m$ be arbitrary. We show that $\msys n
\models \bictlspec$ iff $\msys n \models \varphi[v_1 \mapsto 1, \ldots, v_m
\mapsto m]$. The Lemma then follows.

$\boldsymbol{(\Longleftarrow)}$ For the right to left direction assume that $\msys n \models
\bictlspec$. Then, by the definition of satisfaction of $\bictl$
(Definition~\ref{def:sat}), we have that for all $h: \set{v_1, \ldots, v_m}
\rightarrow \set{1,\ldots,n}$,  it holds that $\msys n \models \varphi[v_1
\mapsto h(v_1), \ldots, v_m \mapsto h(v_m)]$.  Hence, $\msys n \models
\varphi[v_1 \mapsto 1, \ldots, v_m \mapsto m]$.


$\boldsymbol{(\Longrightarrow)}$ For the left to right direction assume that $\msys n \models
\varphi[v_1 \mapsto 1, \ldots, v_m \mapsto m]$. Let $I = \set{i_1, \ldots, i_m}
\in \set{1, \ldots, n}^m$ be an arbitrary set of $m$ distinct integers within
$\set{1,\ldots,n}$.  Consider $\sigma : \set{1, \ldots, n} \rightarrow \set{1,
\ldots, n}$ be a bijective mapping such that $\sigma(j) = i_j$ for all $j$ with
$1 \leq j \leq m$. For a global state $q$, denote by $\pi(q)$ the global state
obtained by replacing every $i$-th local state in $q$ with the $\sigma(i)$-th
local state in $q$. Similarly, for a joint action $\alpha$, denote by
$\pi(\alpha)$ the joint action obtained by replacing every $i$-th local action
in $\alpha$ with the $\sigma(i)$-th local action in $\alpha$. Then, by the
definition of the global transition function
(Definition~\ref{def:globaltransition}), we have that $\globaltr n(q, \alpha) =
q'$  iff $\globalrel n(\pi(q), \pi(\alpha)) = \pi(q')$.  Hence,  $(q, \alpha,
q') \in \globalrel n$ iff $(\pi(q), \pi(\alpha), \pi(q')) \in \globalrel n$.
Therefore, ($\msys n, q) \models  \varphi[v_1 \mapsto 1, \ldots, v_m \mapsto m]$
iff ($\msys n, \pi(q)) \models  \varphi[v_1 \mapsto \pi(1), \ldots, v_m \mapsto
\pi(m)]$. So, by the initial assumption,  it follows that ($\msys n,
\pi(\globalinit{n})) \models  \varphi[v_1 \mapsto \pi(1), \ldots, v_m \mapsto
\pi(m)]$. As $\pi(\globalinit{n}) = \globalinit{n}$, we have that ($\msys n,
\globalinit{n}) \models  \varphi[v_1 \mapsto \pi(1), \ldots, v_m \mapsto
\pi(m)]$. Since the set of indices $I$ was arbitrary, we conclude that $\msys n
\models \bictlspec$.  \end{proof}


\subsection*{Formal definition of the abstract global transition function} In the
following we give the full formal definition of the abstract transition function.
\begin{definition}
  \label{def:globaltransition}
  The {\em global transition function} $\globaltr{m}_{zo} : \globalstates{m}_{zo}
  \times \globalacts{m}_{zo} \rightarrow \globalstates{m}_{zo}$ of the zero-one
  system $\sys{m}_{zo}$ satisfies $\globaltr{m}_{zo}(q, \alpha) = q'$ iff the
  following hold:
  \begin{itemize}
    \item $\la{e}{\alpha} \in \prot{e}({\ls{e}{q}})$ and $\tr e(\ls{e}{q},
    \la{e}{q}, A) = q'$, where
    $A = \set{\la{a}{q} \mid a \in \set{1,\ldots,n}} \cup
\la{zo}{q}$.
    \item For all $a \in \set{1,\ldots,m, \it{zo}}$, we have that
    $\la{a}{\alpha} \in \prot{a}(\ls{a}{q})$, $\tr a (\ls{a}{q}, \la{i}{\alpha},
    A, \la{e}{\alpha}) = \lprv{a}{q'}$, where  $A = \set{\la{a}{q} \mid a \in
    \set{1,\ldots,n}} \cup \la{zo}{q}$, and $\obs
    a((\lprv{a}{q'},\lper{a}{q}),\ls{e}{q'}) = \lper{a}{q'}$.
  \end{itemize}
\end{definition}





\subsection*{Proof of Theorem 1}
\begin{customthm}{1}
$\msys n \leq_b \masys m$ for any $n \geq m+1$ and $b \in \mathbb N$.
\end{customthm}
\begin{proof}
Let $n \geq m+1$. We show that $\msys n
\leq_b \masys m$ for any $b \geq 0$.
 Define $\gamma_n
\colon \globalstates n \rightarrow \aglobalstates m$ to map concrete states in
$\msys n$ to abstract states in $\masys m$ as follows:
\begin{align*}
  \gamma_n(q) =  \langle &\ls{1}{q}, \ldots, \ls{m}{q}, \\
  &\set{\ls{i}{q} \mid i \in \set{m + 1, \ldots, n}}, \\
  &\ls{e}{q}  \rangle.
\end{align*}

We show by induction on $b$ that $\sim_b = \set{(q, \gamma_n(q)) \mid q \in
\globalstates n }$ is a $b$-bounded simulation relation between $\msys n$ and
$\masys m$. 


{\bf Base step.} For the base step, let $b = 0$ and $(q, \abstate) \in \sim_0$
be arbitrary. As $\gamma_n$ preserves the local states of the agents $1, \ldots,
m$, we have that $q \in \valuation n(p, a)$ implies that $\abstate \in
\avaluation(p, a)$ for any $p \in \atprop$ and $a \in \set{1, \ldots, m}$, thus
$\sim_0$ is $0$-bounded simulation between $\msys n$ and $\masys m$.  

{\bf Inductive step.} For the inductive step assume that for any $b >0$,
$\sim_b$ is a $b$-bounded simulation between $\msys n$ and $\masys m$. We show
that $\sim_{b+1}$ is a $(b+1)$-bounded simulation relation  between $\msys n$
and $\masys m$.  As $\sim_{b-1} = \sim_0$, $\sim_{b-1}$ satisfies the first
condition a condition of a bounded simulation relation. To show that it
satisfies the second condition, let $(q, \abstate) \in sim_{b+1}$ be arbitrary
and assume that $(q, \alpha, q') \in \globalrel n$ for some joint action
$\alpha$ and global state $q'$. We need to show that there is an abstract joint
action $\abaction$ and an abstract global state $\abstatep$ such that
$(\abstate, \abaction, \abstatep) \in \aglobalrel m$ and $(q', \abstatep) \in
\sim_b$.  Define $\delta_n \colon \globalacts n \rightarrow \aglobalacts m$ to
map joint actions in $\msys n$ to joint actions in $\masys m$ as follows:
\begin{align*}
  \delta_n(\alpha) =  \langle &\la{1}{q}, \ldots, \la{m}{q}, \\
  &\set{\la{i}{q} \mid i \in \set{m + 1, \ldots, n}}, \\
  &\la{e}{q}  \rangle.
\end{align*}
Let $\abaction = \delta_n(\alpha)$. By the definition of the abstract protocol
function we have that $\abaction \in \prot{zo}(\abstate)$. So $(\abstate,
\abaction, \abstatep) \in \aglobalrel m$, where $\abstatep = \aglobaltr m
(\abstate, \abaction)$. By the definition of the concrete and abstract local
transition functions, and since the set $\set{\la{i}{\alpha} \mid i \in
\set{1,\ldots,n}}$ of actions in $\alpha$ equals the set $\set{\la{i}{q} \mid i
\in \set{1,\ldots,m}} \cup \set{\alpha_t \mid \exists l_t \colon (l_t, \alpha_t)
\in \la{zo}{\abaction}}$ of actions in $\abaction$, we have that $\ls{i}{q'} =
\ls{i}{\abstate}$ for $i \in \set{1,\ldots,m}$, and $\set{\ls{i}{q'} \mid i \in
\set{m+1,\ldots,n}} = \ls{zo}{\abstate'}$.  Therefore $(q', \abstatep) \in
\sim_b$ as required.

We have thus proven that for any $b \geq 0$, $\sim_b = \set{(q, \gamma_n(q))
\mid q \in \globalstates n }$ is a $b$-bounded simulation relation between
$\msys n$ and $\masys m$.  As by the definitions of the initial global states of
abstract and concrete models, we have that $(\globalinit{n}, \aglobalinit{,})
\in \sim_b$ we have that $\msys b \leq b \masys n$ for any $b \geq 0$.

\end{proof}


\subsection*{Proof of Theorem 2}
\begin{customthm}{2}
There is $n \geq m+1$ such that $\masys m \leq_b \msys n$ for any $b \geq 0$.
\end{customthm}

\begin{proof}

The proof of the Theorem is by induction on~$b$. For $n \geq m+1$,  define
$\gamma_n \colon \globalstates n \rightarrow \aglobalstates m$ to map concrete
states in $\msys n$ to abstract states in $\masys m$ as follows:
\begin{align*}
  \gamma_n(q) =  \langle &\ls{1}{q}, \ldots, \ls{m}{q}, \\
  &\set{\ls{i}{q} \mid i \in \set{m + 1, \ldots, n}}, \\
  &\ls{e}{q}  \rangle.
\end{align*}

{\bf Base step.} For the base step, let $b = 0$ and $n = m+1$. Define $$\sim_0 =
\set{(\abstate, q) \mid \abstate \in \aglobalstates{m}, q \in
\aglobalstates{m+1}, \gamma_{m+1}(q) = \abstate}.$$ As $\gamma_{m+1}$ preserves
the local states of the agents $1, \ldots, m$, we have that $\abstate \in
\avaluation m(p, a)$ implies that $q \in \valuation{m+1}(p, a)$ for any $p \in
\atprop$ and $a \in \set{1, \ldots, m}$, thus  $\sim_0$ is $0$-bounded
simulation between $\masys m$ and $\msys{m+1}$. Therefore, since
$(\aglobalinit{m}, \globalinit{m+1}) \in \sim_0$, we what that $\masys m \leq_0
\msys n$. 

{\bf Inductive step.} For the inductive step, assume that for each $i \in
\set{1, \ldots, b}$ there is $n \geq m + 1$ such that $\masys m \leq_b \msys n$
by means of relation $\sim_i$. We show that there is $n' \geq n$ such that
$\masys m \leq_{b+1} \msys{n'}$. Let $|\prot t| = \max \set{|\prot t(l_t)| \mid
l_t \in \lstates{t}}$ and  $n' = m + n * |\prot t|$.  Define the following
relations $\sim_i \subseteq \aglobalstates{m} \times \globalstates{n'}$ between
the abstract states in $\aglobalstates{m}$ and the global states in $\msys{n'}$:

\begin{itemize}

  \item  $\sim'_0 = \set{(\abstate, q) \mid  q_{ab \rightarrow m} =
  \projection{q}{m} [m]}$, where $q_{ab \rightarrow m}$ and $\projection{q}{n}$
  is the projection of $q_{ab}$ and $q$ onto the first~$m$ agents, e.g.,
  $\projection{q}{m} = \tuple{\ls{1}{q}, \ldots, \ls{m}{q}, \ls{e}{q}}$.


  \item for $i \in \set{1, \ldots, b + 1}$, we have that $(\abstate, q) \in
  \sim'_i$ if:
  \begin{itemize}
    \item  $(\abstate,\projection{q}{n}) \in \sim_{i-1}$, and 
    \item  for $j \in [n]$ and $k \in [|\prot t|]$, 
    $\ls{m + (j - 1) * |\prot t| + k}{q} = \ls{j}{q}$.
  \end{itemize}
\end{itemize}

We show that for every $x \in \set{0, \ldots, b+1}$, $\sim'_x$ is an $x$-bounded
simulation between $\masys m$ and $\msys{n'}$. We have three cases.

\begin{itemize}
  \item Case 1: $x=0$.  Since  $q_{ab \rightarrow m} = \projection{q}{m}$, we have that
  $q \in \avaluation m(p, a)$ implies that $q \in \valuation n (p, a)$ for any
  $p \in \atprop$ and $a \in [m]$, thus $\sim'_0$ is a $0$-bounded simulation
  between $\masys m$ and $\msys{n'}$.


  \item Case 2: $x = 1$.  By the definition of $\sim'_1$ we have
  that $(\abstate, \projection{q}{n}) \in \sim_{0}$, 
  thus  $q \in \avaluation m(p, a)$ implies that $\projection{q}{n} \in
  \valuation n (p, a)$ for any $p \in \atprop$ and $a \in [m]$. Hence, by the
  definition of the valuation function, we have that $q_{ab \rightarrow m} =
  \projection{q}{m}$, thus $(\abstate, q') \in \sim'_0$, so $\sim'_1$
  satisfies the first condition of bounded simulation. 


  To show that it satisfies the second condition, assume that $(\abstate,
  \abaction, \abstatep) \in \aglobalrel m$ for some abstract joint action
  $\abaction$ and abstract global state $\abstatep$. We need to show that there
  is a joint action $\alpha$ and a  global state $q'$ such that $(q, \alpha, q')
  \in \globalrel{n'}$ and $(\abstatep, q') \in \sim'_0$. Define $\alpha$ as
  follows:

  \begin{itemize}
    \item $\la{i}{\alpha} = \la{i}{\abaction}$, for $i \in [m]$.
    \item  for $i \in [n]$, consider the set $\set{a_t \mid (\ls{m + (i - 1) *
    |\prot t| + 1} {q},\alpha_t) \in \la{zo}{\abaction}}$ of template actions
    that are paired with the template state of the $(m + (i-1) + 1)$-the agent
    in the action of the zero-one agent. Assume an ordering $\alpha_1, \ldots,
    \alpha_j$ of this set and define
    \begin{itemize}
    \item  $\la{m + (i -1) * |\prot t| + k}{\alpha} = \alpha_k$  for $k \in [j]$.
    \item $\la{m + (i-1) * |\prot t| + k}{\alpha} = \alpha_1$ for $k \in
    \set{j+1, \ldots, |\prot t|}$.
    \end{itemize}
  \end{itemize}
  By the definition of $\alpha$ we have that $\la{i}{\alpha} \in \prot
  i(\ls{i}{q})$ for every $i \in [n']$. So $(q, \alpha, q') \in \globalrel
  {n'}$, where $q' = \globaltr{n'} (q, \alpha)$. By the definition of the
  concrete and abstract local transition functions, and since the set
  $\set{\la{i}{\alpha} \mid i \in [n']}$ of actions in $\alpha$ equals the set
  $\set{\la{i}{q} \mid i \in \set{1,\ldots,m}} \cup \set{\alpha_t \mid \exists
  l_t \colon (l_t, \alpha_t) \in \la{zo}{\abaction}}$ of actions in $\abaction$,
  we have that $q_{ab \rightarrow m} = \projection{q}{m}$, therefore
  $(\abstatep, q') \in \sim'_0$, as required. 


  \item Case 3: $x \in \set{2, \ldots, b+1}$.   By the definition of $\sim'_x$
  we have that $(\abstate, \projection{q}{n}) \in \sim_{x-1}$. By the definition
  of bounded simulation it follows that $(\abstate, \projection{q}{n}) \in
  \sim_0$, thus  $q \in \avaluation m(p, a)$ implies that $\projection{q}{n} \in
  \valuation n (p, a)$ for any $p \in \atprop$ and $a \in [m]$. Hence, by the
  definition of the valuation function, we have that $q_{ab \rightarrow m} =
  \projection{q}{m}$, thus $(\abstate, q') \in \sim'_0$, so $\sim'_{x}$
  satisfies the first condition of bounded simulation. 


  To show that it satisfies the second condition, assume that $(\abstate,
  \abaction, \abstatep) \in \aglobalrel m$ for some abstract joint action
  $\abaction$ and abstract global state $\abstatep$. We need to show that there
  is a joint action $\alpha$ and a  global state $q'$ such that $(q, \alpha, q')
  \in \globalrel{n'}$ and $(\abstatep, q') \in \sim'_{x-1}$. Since $(\abstate,
  \projection{q}{n}) \in \sim_{x-1}$, we have that there is a joint action
  $\beta$ and a global state $r$ such that $(\projection{q}{n}, \beta, r) \in
  \globalrel n$ and $(\abstate, r) \in \sim_{x-2}$. Define a joint action
  $\alpha \in \globalacts{n'}$ such that
  
  \begin{itemize}
    \item for $i \in [n]$, $\la{i}{\alpha} = \la{i}{\beta}$, 
    \item  for $i \in \set{0, \ldots, n-1}$ and $j \in [|\prot t| -  1]$, $\la{n +
    i * |\prot t| + j}{\alpha} = \la{i}{\beta}$.
  \end{itemize}

  By the definition of $\alpha$, we have that $(q, \alpha, q')  \in
  \globalrel{n'}$ for a global state $q'$ that satisfies

  \begin{itemize}
    \item  $(\abstate,\projection{q'}{n}) \in \sim_{b-1}$, and 
    \item  for $j \in \set{0, \ldots, n-1}$ and $k \in [|\prot t| -  1]$, 
    $\ls{n + j * |\prot t| + k}{q'} = \ls{j}{q'}$.
  \end{itemize}

  It follows that $(q, \alpha, q') \in \sim'_{x-1}$. Consequently, $\sim'_{x}$
  is an $x$-bounded simulation between $\masys m$ and $\msys{n'}$.

\end{itemize}

We have thus proven that for every $x \in \set{0, \ldots, b+1}$, $\sim'_x$ is an
$x$-bounded simulation between $\masys m$ and $\msys{n'}$.  As by the
definitions of the initial global states of abstract and concrete models, we
have that $(\aglobalinit{m}, \globalinit{n'}) \in \sim'_{b+1}$ we have that
$\masys m \leq_{b+1} \msys{n'}$.

\end{proof}



% We say that $\msys n$ and $\masys m$ are {\em $b$-bounded bisimiulation
% equivalent} models if there is a $b$-bounded bisimiulation relation $\sim_b$
% between $\msys n$ and $\masys n$  such that $(\globalinit n, \aglobalinit m) \in
% \sim_b$.

% where $\projection{q}{n} =
%   \tuple{\ls{1}{q}, \ldots, \ls{n}{q}, \ls{e}{q}}$ is the projection of $q$ to
%   the first~$n$ agents,  and 



% \begin{lemma} 
%   \label{lemma:lifting}
%   Let $m \in \mathbb{N}$, $n \geq m$ and $n' > n$. If $\sim_b = \set{\tuple{q,
%   \gamma_n(q)} \mid q \in \globalstates n}$ is a $b$-bounded bisimulation
%   relation between   $\msys n$ and $\masys{m}$, then the relation 
%   $$
%   \sim'_b = \set{(q, \gamma_{n'}(q)) \mid q \in \globalstates{n'}, 
%   \exists r \in \globalstates{n} \colon (r, \gamma_{n'}(q)) \in \sim_b}
%  $$
%   % \subseteq \globalstates{n'} \times \aglobalstates{m}$ satisfying $$(q,
%   % \abstate) \in \sim'_b \text{ iff } (\projection{q}{n}, \abstate) \in \sim_b$$ 
%   is a $b$-bounded bisimulation relation between $\msys{n'}$ and $\masys{m}$.
% \end{lemma}
% \begin{proof}

% Assume that $\sim_b$ is a $b$-bounded bisimulation relation between  $\msys n$
% and $\masys{m}$. We show that the relation $\sim'_b$ is a $b$-bounded
% bisimulation relation between  $\msys{n'}$ and $\masys{m}$. Let $(q, \abstate)
% \in \sim'_b$ be arbitrary. We need to show that $(q, \abstate)$ satisfies
% conditions 1-3 of Definition~\ref{def:bounded-bisimulation}.   By the definition
% of $\sim'$ there is a global state $r \in \globalstates{n}$ such that $(r,
% \gamma_{n'}(q)) \in \sim_b$.



% \begin{itemize}

%   \item[] Condition 1. Assume that $q \in \valuation{n'}(p,a)$ for some $p \in
%   \atprop$, $a \in \set{1, \ldots, m}$.  As  $\abstate = \gamma_{n'}(q)$ and
%   $\gamma_{n'}$ preserves the local states of the agents $1, \ldots, m$, we have
%   that $q \in \valuation{n'}(p,a)$ iff  $\abstate \in \avaluation m(p, a)$.

%   \item[] Condition 2. Assume that $(q, \alpha, q') \in \globalrel{n'}$ for a
%   joint action $\alpha$ and global state $q'$.  We need to show that there is  a
%   joint abstract action $\abaction$ and an abstract global state $\abstatep$
%   such that $(\abstate, \abaction, \abstatep) \in \aglobalrel m$ and $(q',
%   \abstatep) \in \sim'_{b-1}$.  
  
%   Let $\abaction = \delta_{n'}(\alpha)$. By the definition of the abstract
%   protocol function we have that $\abaction \in \prot{zo}(\abstate)$. So
%   $(\abstate, \abaction, \abstatep) \in \aglobalrel m$, where $\abstatep =
%   \aglobaltr m (\abstate, \abaction)$. By the definition of the concrete and
%   abstract local transition functions, and since the set $\set{\la{i}{\alpha}
%   \mid i \in \set{1,\ldots,n'}}$ of actions in $\alpha$ equals the set
%   $\set{\la{i}{q} \mid i \in \set{1,\ldots,m}} \cup \set{\alpha_t \mid \exists
%   l_t \colon (l_t, \alpha_t) \in \la{zo}{\abaction}}$ of actions in $\abaction$,
%   we have that $\ls{i}{q'} = \ls{i}{\abstate}$ for $i \in \set{1,\ldots,m}$, and
%   $\set{\ls{i}{q'} \mid i \in \set{m+1,\ldots,n'}} = \ls{zo}{\abstate'}$.
%   Therefore $\gamma_{n'}(q') =  \abstatep$. 

%   Now as $(r, \abstate) \in \sim_b$, there is a joint action $\beta$ such that
%   $(r, \beta, r') \in \globalrel{n}$ and $(r', \abstatep) \in \sim_{b-1}$. As
%   $\abstatep = \gamma_{n'}(q')$, we have that $(q', \abstatep) \in \sim'_{b-1}$,
%   as required.

%   \item Condition 3. Assume that $(\abstate, \abaction, \abstatep) \in
%   \aglobalrel m$ for an abstract joint action $\abaction$ and abstract global
%   state $\abstatep$. We need to show that there is a joint action $\alpha$ and a
%   global state $q'$ such that $(q, \alpha, q') \in \globalrel{n'}$ and $(q',
%   \abstatep) \in \sim'_{b-1}$. 

%   Since $(r, \abstate) \in \sim_b$, there is a joint action $\beta$ such that
%   $(r, \beta, r') \in \globalrel{n}$ and $(r', \abstatep) \in \sim_{b-1}$.
%   Construct a joint action $\alpha$ in $\msys{n'}$ such that 
  
%   \begin{itemize}
%     \item for each $i \in \set{1,\ldots,n}$, $\la{i}{\alpha} = \la{i}{\beta}$;
%     \item for each $i \in \set{n+1, \ldots, n'}$,  $\la{i}{\alpha} =
%     \la{j}{\beta}$, where $j$ is the smallest number in $\set{m + 1,\ldots,n}$, for
%     which $\ls{i}{q} = \ls{j}{r}$.
%   \end{itemize}
  
%   As $\gamma_{n'}(q) = \gamma_{n}(r)$ we have that $\la{i}{\alpha} \in
%   \prot{i}{q}$ for $i \in \set{1,\ldots,n'}$. So $(q, \alpha, q') \in
%   \globalrel{n}$, where $q' = \globaltr{n'}(q, \alpha)$.  By the definition of
%   the concrete local transition functions, and since the set
%   $\set{\la{i}{\alpha} \mid i \in \set{m,\ldots,n'}}$ of actions in $\alpha$
%   equals the set $\set{\la{i}{q} \mid i \in \set{m + 1,\ldots,n}}$ of actions in
%   $\beta$, we have that $\gamma_{n'}{q'} = \gamma_n(r')$. Therefore, $(r',
%   \gamma_{n'}(q')) \in \sim_{b-1}$, as required.

% \end{itemize}
% \end{proof}

% $l_t \rightarrow_\rho l'_t$ if $l_t$ if there is a sequence
% $(l_1, \alpha_1),  \ldots, (l_{x-1}, \alpha_{x-1}), l_x$, where 
% \begin{itemize}
%   \item $x = \it{len}(\rho)$, $l_1 = l_t$, $l_x = l'_t$,
%   \item for each $i \in [\it{len}(\rho) -1]$, we have that 
%   \begin{itemize}
%     \item $l_i = (\prv i, \per i)$,
%     \item $(l_i, \alpha_i) \in \ls{zo}{\rho(i, \it{act})}$,
%     \item $\tr{t}(l_i, \alpha_i, A, \la{e}{\rho(i, \it{act})}) = \prv {i + 1}$ and
%     \item $\obs t(\prv{i+1}, \per i, \ls{e}{\rho(i+1)}) = \per{i+1}$.
%   \end{itemize}
% \end{itemize}
% 
% $A = \set{\la_a(\rho(x, \it{act}) \id a \in set{1, \ldots,m}} \cup \la{zo}(\rho(x)$
% $$
% \max_{i \in [b], \rho \in \Pi(q)}  \left( \sum_{l_t \dashrightarrow_{\rho[i,b]} l'_t} \#\set{a_t
% \mid (l_t,a_t) \in \la{zo}{\rho(i,\it{act})}} \right)
% $$




% \globalstates{m+1})}$ is a $0$-bisimulation relation between $\msys{m+1}$ and
% $\masys m$ are $0$-bisimulation equivalent. Clearly, we have that
% $\tuple{\globalinit{m+1}, \aglobalinit{m}} \in sim_0$.  As  $\gamma_{m+1}$
% preserves the local states of the agents $1, \ldots, m$, we also have that
% $q \in \valuation{n'}(p,a)$ iff  $\abstate \in \avaluation m(p, a)$, therefore
% $\sim_0$ is a $0$-bisimulation relation between $\msys{m+1}$ and $\masys m$. 

% \end{proof}









% \begin{proof}
% We first prove the left to right direction and then the right to left direction.

% $\boldsymbol{(\Longrightarrow)}$ For the left to right direction assume that  $\masys m \models
% \varphi[v_1 \mapsto 1, \ldots, v_m \mapsto m]$.  To establish the preservation
% of $\bictl$ formulae from the abstract model to the concrete models we show that
% $\masys m$ simulates every concrete model $\msys n$ for $n \geq m$. Intuitively
% a model simulates another if every behaviour exhibited by the simulated model is
% also exhibited by the simulating model.  Since a $\bictl$ formula may only
% quantify over all paths, every formula satisfied by the simulating model is also
% satisfied by the simulated model.  Formally,  a relation $\sim \subseteq
% \globalstates n \times \aglobalstates m$ is a simulation between $\msys n$ and
% $\masys m$ if the following conditions hold:
% \begin{enumerate}
%   \item $(\globalinit n, \aglobalinit m) \in \sim$.
%   \item Whenever $(q, \abstate) \in \sim$, then
%   \begin{enumerate}
%     \item If $q \in \valuation n(p,a)$, for some $p \in \atprop$ and $a \in
%     \set{1,\ldots,m}$, then $\abstate \in \avaluation m(p, a)$.
%     \item If $(q, \alpha, q') \in \globalrel n$ for a joint action $\alpha$ and
%     global state $q'$, then there is an abstract joint action
%     $\abaction$ and an abstract global state $\abstatep$ such that
%     $(\abstate, \abaction, \abstatep) \in \aglobalrel m$ and
%     $(q', \abstatep) \in \sim$. 
%     \end{enumerate}
% \end{enumerate}
% If there is a simulation relation between $\msys n$ and $\masys n$, then we say
% that $\masys m$ simulates $\msys n$. If $\masys m$ simulates $\msys n$, then
% $\masys n \models \varphi[v_1 \mapsto 1, \ldots, v_m \mapsto m]$ implies that
% $\msys n \models \varphi[v_1 \mapsto 1, \ldots, v_m \mapsto
% m]$~\cite{ClarkeGrumbergLong94} (the result in the cited work is shown w.r.t. 
% arbitrary Kripke structures and unbounded $\ctl$ formulae).

% In the following we show that $\masys m$ simulates $\msys n$. Define $\gamma_n
% \colon \globalstates n \rightarrow \aglobalstates m$ to map concrete states in
% $\msys n$ to abstract states in $\masys m$ as follows:
% \begin{align*}
%   \gamma_n(q) =  \langle &\ls{1}{q}, \ldots, \ls{m}{q}, \\
%   &\set{\ls{i}{q} \mid i \in \set{m + 1, \ldots, n}}, \\
%   &\ls{e}{q}  \rangle.
% \end{align*}
% We show that $\sim = \set{(q, \gamma_n(q)) \mid q \in \globalstates n }$ is a
% simulation relation between $\msys n$ and $\masys m$. By the definitions of the
% initial global states of abstract and concrete models the relation satisfies the
% first condition of the simulation. 

% To show that it satisfies the second condition, let $(q, \abstate) \in \sim$ by
% arbitrary. As $\gamma_n$ preserves the local states of the agents $1, \ldots,
% m$, we have that $q \in \valuation n(p, a)$ iff $\abstate \in \avaluation(p, a)$
% for any $p \in \atprop$ and $a \in \set{1, \ldots, m}$, thus the first clause of
% the second condition is satisfied. 

% For the second clause,  assume that $(q,
% \alpha, q') \in \globalrel n$ for some joint action $\alpha$ and global state
% $q'$. We need to show that there is an abstract joint action $\abaction$ and
% an abstract global state $\abstatep$ such that $(\abstate, \abaction, \abstatep) \in
% \aglobalrel m$ and $(q', \abstatep) \in \sim$.  Define $\delta_n \colon
% \globalacts n \rightarrow \aglobalacts m$ to map joint actions in $\msys n$ to
% joint actions in $\masys m$ as follows:
% \begin{align*}
%   \delta_n(\alpha) =  \langle &\la{1}{q}, \ldots, \la{m}{q}, \\
%   &\set{\la{i}{q} \mid i \in \set{m + 1, \ldots, n}}, \\
%   &\la{e}{q}  \rangle.
% \end{align*}
% Let $\abaction = \delta_n(\alpha)$. By the definition of the abstract protocol
% function we have that $\abaction \in \prot{zo}(\abstate)$. So $(\abstate,
% \abaction, \abstatep) \in \aglobalrel m$, where $\abstatep = \aglobaltr m
% (\abstate, \abaction)$. By the definition of the concrete and abstract local
% transition functions, and since the set $\set{\la{i}{\alpha} \mid i \in
% \set{1,\ldots,n}}$ of actions in $\alpha$ equals the set $\set{\la{i}{q} \mid i
% \in \set{1,\ldots,m}} \cup \set{\alpha_t \mid \exists l_t \colon (l_t, \alpha_t)
% \in \la{zo}{\abaction}}$ of actions in $\abaction$, we have that $\ls{i}{q'} =
% \ls{i}{\abstate}$ for $i \in \set{1,\ldots,m}$, and $\set{\ls{i}{q'} \mid i \in
% \set{m+1,\ldots,n}} = \ls{zo}{\abstate'}$. Therefore $(q', \abstatep) \in \sim$
% as required.



% $\boldsymbol{(\Longleftarrow)}$  For the right to left direction  assume that
% $\pnis \models  \varphi[v_1 \mapsto 1, \ldots, v_m \mapsto m]$. For
% contradiction assume that $\masys m \not \models \varphi[v_1 \mapsto 1, \ldots,
% v_m \mapsto m]$.  Then, there is a finite path $\abpath$
% of evidencing the dissatisfaction of the formula. 

% We construct by induction on the length $len(\abpath)$ of $\abpath$ a concrete
% path $\rho$ in $\masys n$, $n \geq m$, such that $\gamma_n(\abpath(j)) =
% \rho(j)$ for $j \in \set{1,\ldots,len(\abpath)}$.  For the base case of
% $len(\abpath)=1$, let $\rho = \globalinit n$ for any $n > m$.  We have that
% $\gamma_n{\abpath(1)} = \rho{1}$, thus the
% base case is satisfied.

% For the inductive case assume that $len(\rho) = x$ and that we have constructed
% a concrete path $\rho$ in $\masys n$, $n > m$, of the same length such that
% $\gamma_n(\abpath(j)) = \rho(j)$ for $j \in \set{1,\ldots,x}$. We show that a
% similar path can be constructed for $len(\abpath) = x+1$.  

% Let $\abaction$ be the joint abstract action for which $(\abpath(l), \abaction,
% \abpath(x+1)) \in \aglobalrel m$. For any template local state $l_t$ let
% $\#(\abaction, l_t)$ denote the number of state-action pairs involving $l_t$ in
% $\la{zo}{\abaction}$, i.e., $\#(\abaction, l_t) = \set{\alpha_t \mid (l_t,
% \alpha_t) \in \la{zo}{\abaction}}$. Consider $\it{diff}(l_t)$ to be the
% difference between $\#(\abaction, l_t)$ and the number of agents from $\set{m+1,
% \ldots, n}$ that are in local state $l_t$ in $\rho(x)$, i.e.,  $\it{diff}(l_t) =
% \#(\abaction, l_t) - \#\set{i \mid i \in \set{m+1,\ldots,m} \text{ and }
% \ls{i}{\rho(x)} = l_t}$. 


% Let $n' = \sum_{l_t} \max( \it{diff}(l_t), 0) + n$.  We construct a concrete
% path $\rho'$ in $\msys{n'}$ by specifying the sequence of actions of each of the
% agents.  The first $n$ agents perform the exact sequence of actions of the
% agents in $\masys n$. For each $l_t$ with $\it{diff}(l_t) > 0$, the next
% $\it{diff}(l_t)$ agents perform the same sequence of actions as the first agent
% that is in state $l_t$ in $\rho(x)$ (by the inductive hypothesis there is at
% least one agent in every $l_t$ with $\#(\abaction, l_t) \geq 1$). By
% construction the agents $\set{m+1, \ldots, n'}$ in $\rho'(x)$ can collectively
% perform the set of actions $\la{zo}{\abpath(x)}$ performed by the zero-one agent
% in $\abpath(x)$. This transitions the system to a state $\rho'(x+1)$ such that
% $\gamma_n(\abpath(x+1)) = \rho'(x+1)$ as required.

% We therefore have that $\msys n'  \not \models \varphi[v_1 \mapsto 1, \ldots,
% v_m \mapsto m]$, thus $\pnis \not \models \varphi[v_1 \mapsto 1, \ldots,
% v_m \mapsto m]$, concluding the Theorem.

% \end{proof}




\subsection*{Training of a neural observation function for the Guarding Game} 

Here we provide more details regarding the training of the neural observation
function for the guarding game.

We used deep Q-learning, a type of reinforcement learning (RL) algorithm.
During the training, the game was played by 4 agents (in the main paper
mistakenly specified as 3), and the game parameters were set as $M_h = 4$,
$G_r = -2$, $R_r = 1$ and $U_r = -3$.
% The rewards were assigned to reflect the tension between individual and
% collective interests.
%
At the beginning, the agents' health was $M_h$.
%
At each turn of the game, the agents that were alive could decide whether they
volunteer to guard or not. When there were volunteers, a seperate arbitration
function decided randomly how many of the volunteers undertook the guarding
action (the other volunteers took the resting action). Each agent had a 50\%
chance of being selected if it volunteered, but it was impossible for no agent
to be selected as long as there were volunteers.

The reward function for each agent consisted of
\begin{inparaenum}[\it (i)]
\item  a social reward,
\item an individual reward,
\item desire to personally live, and
\item fear of collective failure.
\end{inparaenum}
The social reward for an agent was expressed as the average change in every
other agent's health, minus the fraction of agents who are dead at the end of
the turn, divided by two (to provide an arithmetic average of the two
values). This was scaled by the agent's current normalised health. The
individual reward was expressed as the change in the agent's health. The desire
to personally live was expressed as plus one point if the agent is alive at the
end of the turn and minus one if it is not, and the fear of collective failure
as minus two if all agents are dead at the end of the turn.



All agents shared the same neural network, creating a system in which the
agents were learning to play against exact copies of themselves.
%
We used an experience replay buffer and a separate, periodically updated target
network for providing estimations.
%
Each time the policy network was trained, a random minibatch of experiences
(tuples consisting of a state transition and associated reward) was selected
from the replay buffer as training data.
%
Each experience's starting state is then input into the \emph{policy network}
to produce an array of target Q-values for each action. The estimate for
the action which was taken is then modified with the \emph{target network's}
estimate of the \emph{value of the most valuable action in the next state},
multiplied by the discount factor. This produces, for each experience and
therefore starting state in the minibatch, Q-value targets for each action.
%
This way, the Q-value of each action is stabilised, as the estimated value of
continuing to follow the policy is always produced by an `old' copy of the
policy net.

The reinforcement learning procedure ran for 1024 episodes of a maximum length
of 16 turns each, if the agents did not all expire. There were 4 agents in the
training simulation. The target network was updated every 32 turns, the replay
buffer held 4096 experiences and was sampled in minibatches of size 16.

The network contained two hidden layers each four units wide and using ReLU
activation functions. The output layer used linear activation and had two units
corresponding to the two actions (rest and guard) available while the agent had
not expired. We used RMSProp gradient descent in minibatches corresponding to
those described in the process of experience replay; the batch size was set to
the number of experiences in each replay sample. The loss function used was
Huber loss, and the actions taken during training utilised an epsilon-greedy
exploration schedule \cite{Mnih+15,HaaseltGS16}.
% This is as per the original deep Q-learning paper \cite{Mnih+15} and the
% follow up proposing double deep Q-learning, \cite{HaaseltGS16}.
%
For RMSProp, a learning rate of 0.0025 was used, with the default $\rho$ of 0.9
and momentum of 0. For epsilon-greedy exploration, the starting epsilon was 1.0
and it decayed by a factor of 0.999 every turn to a minimum of 0.01. We built
and trained the network using the TensorFlow Keras library.

%%% Local Variables:
%%% mode: latex
%%% fill-column: 79
%%% TeX-master: "../main"
%%% End:
