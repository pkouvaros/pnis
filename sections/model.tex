Interpreted systems are a standard semantics for describing multi-agent systems
[Fagin et al., 1995b]. They provide a natural setup to interpret specifications
in a variety of languages including temporal-epistemic logic and alternating
temporal logic [Fagin et al., 1995a; Lomuscio and Raimondi, 2006].
Parameterised interpreted systems is a parametric extension of interpreted
systems put forward to reason about unbounded multi-agent
systems~\cite{KouvarosLomuscio15b}. The parameter in a system of this kind
denotes the number of agents composing the system, each homogeneously
constructed from an agent template. We here extend parameterised interpreted
systems to Parameterised Neural-symbolic Interpreted Systems (PNIS), where the
template for the agents is not purely symbolic but it comprises a perception
mechanism that is implemented via neural networks and which is coupled with a
symbolic action mechanism. This neural-symbolic treatment of the agents follows
the Neural Interpreted Systems (NIS) model from~\cite{Akintunde+20b}.
Differently from NPIS however, NIS are limited to standard non-parametric
systems with a pre-defined number of agents.

A PNIS consists of the descriptions of an agent template, from which an
unbounded number of concrete agents may be constructed, and an environment in
which the agents operate. The agent template is defined by the following. 

\begin{definition}[Agent template.]
An {\em agent template} is a tuple $\ag t = \tuple{\lstates t, \init t, \obs t,
\acts t, \prot t, \tr t }$, where:
\begin{itemize}
  \item $\lstates t = \prvs t \times \pers t$ is a nonempty (possibly infinite)
  set of local states.  Each local state  is a pair $\tuple{\prv t, \per t}$ of
  a private state $\prv t \in \prvs t \subseteq \mathbb R^{m_{prv}}$ and a
  percept $\per t \in \pers t \subseteq \mathbb R^{m_{per}}$ that encodes the
  perception a agent has about the environment.  

  \item $\init t \in \lstates t$ is the unique initial template state.
  
  \item $\obs t : \lstates t \times \lstates e \rightarrow \pers t$ is an
  observation function that maps pairs of template local states $\lstates t
  \subseteq \mathbb{R}^{m_{prv} + m_{per}}$ and environment states $\lstates e
  \subseteq \mathbb{R}^{m_e}$ (defined below) to percepts $\pers t \subseteq
  \mathbb{R}^{m_{per}}$. The observation function is implemented via a PWL FFNN $f :
  \mathbb{R}^{m_{prv} + m_{per} + m_e} \rightarrow \mathbb{R}^{m_{per}}$.

  \item $\acts t$ is a nonempty and finite set of actions.
  
  \item $\prot t : \lstates t \rightarrow 2^{\acts t} \setminus \set{\emptyset}$
  is a local protocol function that selects which actions may be performed at a
  given local state.%%\nb{E: shouldn't the protocol take as input $\pers{t}$?}

  \item $\tr t: \lstates t \times \acts t \times 2^{\acts t \cup \acts e}
  \rightarrow \prvs t$ is a local transition function  that determines the next
  private state for a concrete agent given its current local state, its action
  and the set of actions performed by all the other agents  and the environment.
  
\end{itemize}  
\end{definition}

An environment template has a similar description to an agent template but does
not include a perception mechanism.

\begin{definition}[Agent environment.]
An {\em agent environment} is a tuple $\ag e = \tuple{\lstates e, \acts e, \prot
e, \tr e }$, where $\lstates e  \subseteq \mathbb R^{m_e}$ is a nonempty
(possibly) infinite set of local states, $\acts e$ is a nonempty and finite set
of actions,  $\prot e : 2^{\acts e} \setminus \set{\emptyset}$ is a local
protocol function, and $\tr e : \lstates e \times \acts e \times 2^{\acts t} 
\rightarrow \lstates e$ is a local transition function.
\end{definition}

Agent and environment templates define the main formal structure we will be
using in this paper.

\begin{definition}[Parameterised Neural Interpreted System]
  A {\em Parameterised Neural Interpreted System} (PNIS) is a tuple $\mathcal S
  = \tuple{\ag t, \ag e, \ell}$, where $\tlabel{} : \atprop \rightarrow
  2^{\lstates t}$ is a labelling function on the agent  template states for a set
  $\atprop$ of atomic propositions.
\end{definition}

A PNIS $\mathcal S$ gives a  parametric description of an unbounded collection
of {\em concrete} NIS. In particular, for any value $n \geq 1$ of the parameter
the concrete system $\sys n$ composes~$n$ copies $\ag 1, \ldots, \ag n$ of the
agent template with the environment $\ag e$. We write $\agents n$ for the set
$\agents n = \set{\ag 1, \ldots, \ag n}$ of concrete agents instantiated from $\ag
t$.   A global state $q = \tuple{\lstate 1, \ldots, \lstate n, \lstate e}$ in
$\sys n$ is a tuple of local states for all the agents and the environment in
$\sys n$; it describes the system at a particular instant of time.  For a global
state $g$ we write $\lprv{a}{q}$, $\lper{a}{q}$ and $\ls{a}{q}$ to denote the
private part $\prv a$ and the perception part $\per a$ of the local state
$\ls{a}{g} = \tuple{\prv a, \per a}$ of agent $\ag a$ in $g$.  The set
$\globalstates n = \lstates 1 \times \ldots \times \lstates n \times L_e$ of all
possible global states is the Cartesian product of the agents’ sets of local
states.  A joint action $\alpha = \tuple{\alpha_1, \ldots, \alpha_n, \alpha_e}$
in $\sys n$ is a tuple of local actions for all the agents and the environment.
For a joint action $\alpha$ we write $\la{a}{\alpha}$ to denote the local action
of agent $\ag a$ in $\alpha$.  The set $\globalacts n = \act 1 \times \ldots \act n
\times \act e$ of all possible joint actions is the Cartesian product of the
agents’ sets of local actions.  A concrete system's global states evolve over
time in compliance with the following global transition relation.

% , i.e., for each agent $a \in \set{1,\ldots,n}$ we have that $a =
% \tuple{\lstates a, \init a, \obs a, \acts a, \prot a, \tr a} = \ag t$, 

\begin{definition}[Concrete Neural Interpreted System]
  \label{def:concreteystem}
Given an NPIS $\mathcal S = \tuple{\ag t, \ag e, \ell_t}$ and $n \geq 1$, a
concrete neural interepted system is a tuple $\sys n = \tuple{ \set{\ag 1,
\ldots, \ag n, \ag e}, \globalinit n, \valuation n}$, where $\globalinit n =
\tuple{\init 1, \ldots, \init n, \init e}$ is the initial global state and
$\valuation n :  \atprop \times \set{1, \ldots, n } \rightarrow 2^{\globalstates
n}$ is the concrete labelling function satisfying $q \in \valuation n(p,a)$ iff
$\ls{a}{q} \in \tlabel a(p)$.
\end{definition}

So the atomic propositions in a concrete system are indexed by each of the
concrete agents: $(p, a)$ holds in a global state if the agent $a$ is at a local
state labelled with $p$ by the template labelling function. This will enable us
to construct specifications independently of the size of the concrete system on
which they are evaluated.

Given the current global state $\tuple{\lstate 1, \ldots, \lstate n, \lstate
e}$, where $\lstate a = \tuple{\prv a, \per a}$ for $a \in \agents n$, of the
agents and the environment, the operational cycle of the agents is described as
follows. First, every agent $\ag a \in \agents n \cup \set{\ag e}$ selects an
action $\act a$ that is permitted by its protocol, i.e., $\act a \in \prot
a(\lstate a)$. Then, the agents synchronously perform the selected actions.
Following  the execution of the actions each agent $\ag a \in \agents n$ updates
the private component of its local state as per its local transition function to
$\pprv a = \tr a ( \lstate a, A, \act e)$, where $A = \set{ \act b \mid b \in
\set{1, \ldots, n} \setminus \set{a}}$.  This generates an intermediate local
state $\plstate{a} = \tuple{\pprv a, \per a}$ for each of the agents. Similarly,
the  environment updates its local state to $\plstate e$. Finally, every agent
$\ag a \in \agents n$ observes the update on its local state and the update on the
local state of the environment via its neural perception module $\obs a$, thus
generating a percept $\pper a = \obs a(\plstate a, \plstate e)$, with which it
updates its perception part thereby obtaining a new local state $\pplstate a =
\tuple{ \pprv a, \pper{a}}$.


Differently from the standard treatment of interpreted systems and NIS, the
transition function of a concrete agent in PNIS  does not depend on the joint
action performed in the system, but it depends on the local action performed by
the  agent and on the on the set of actions performed by the rest of the agents
and the environment.  Thus, the identities of the agents are abstracted away in
a joint action, thereby reflecting the unbounded nature of PNIS. In other words,
whereas  a concrete agent can observe which actions were performed in the system
at a given time,  it cannot observe which agent or how many agents performed
each action.

We now formally define the temporal evolution of a concrete system
$\sys n$.



\begin{definition}[Global transition function] 
  \label{def:globaltransition}
  The {\em global transition function} $\globaltr n : \globalstates n
  \times \globalacts n \rightarrow \globalstates n$ of a concrete system $\sys n$
  satisfies $\globaltr n(q, \alpha) = q'$ iff the following hold:
  \begin{itemize}
    \item $\la{e}{\alpha} \in \prot{e}({\ls{e}{q}})$ and $\tr e(\ls{e}{q},
    \la{e}{q}, A) = q'$, where $A = \set{\la{a}{q} \mid a \in
    \set{1,\ldots,n}}$; i.e., the environment’s action is protocol compliant and
    its local state is updated as per its local transition function w.r.t. the
    set of actions that were performed in the round. 

    \item For all $a \in \set{1,\ldots,n}$, we have that $\la{a}{\alpha} \in
    \prot{a}(\ls{a}{q})$, $\tr a (\ls{a}{q}, \la{i}{\alpha}, A) = \lprv{a}{q'}$,
    where $A = \set{\la{b}{q} \mid j \in \set{1,\ldots,n} \setminus \set{a}}$,
    and $\obs a((\lprv{a}{q'},\lper{a}{q}),\ls{e}{q'}) = \lper{a}{q'}$; i.e.,
    the agent's action is protocol compliant, the private part of its local
    state is updated as per its local transition function w.r.t to the set of
    actions performed by the rest of the agents, and the perception part of its
    local state is updated as per its observation function.
  \end{itemize}
\end{definition}


Each concrete system is associated with a temporal model 
that we will use  to interpret our specification language.

\begin{definition}[Model]
  \label{def:model}
  Given a concrete NIS $\sys n$, the induced model of $\sys n$, or simply model,
  is a tuple $\msys  n = \tuple{\globalstates n, \globalacts n, \globalrel n,
  \valuation n}$, where $G(n)$ is the set of global states, $\globalacts n$ is the
  set of joint actions, $\globalrel n$ is the global transition relation defined
  as $(q, \alpha, q') \in \globalrel n$ iff $\globaltr n(q, \alpha) = q'$, and 
  $\valuation n$ is the labelling function as in
  Definition~\ref{def:concreteystem}.
\end{definition}

A path in a model $\msys n$ is an infinite sequence of global states anc joint
actions $q^0\alpha^0q^1\alpha1\ldots$ such that $(q^i,\alpha^i,q^{i+1}) \in
\globalrel n$ for all $i \geq 0$. Given a global state $q$ in $\msys n$ we write
$\paths{q}$ for the set of all paths originating from $q$.
  
We express specifications for PNIS in an indexed and bounded variant of
Computation Tree Logic ($\ctl$), henceforth $\bictl$. The logic (i) introduces
indexed atomic propositions that are quantified over the agents of the concrete
system the formula in question is evaluated; and (ii) permits only the
construction of formulae whose evaluation can be realised on paths of bounded
lengths. The former extends $\ctl$ by allowing the formulation of properties
irrespective of the concrete system on which they are evaluated.  The latter
restrict $\ctl$ following the undecidability verification for unbounded
formulae~\cite{Akintunde+20}.

\begin{definition}
Given a  set $\atprop$ of atomic propositions and a set $\atvar$ of variables,
the $\bictl$  formulae are defined by the following BNF:
\[
  \varphi \;   ::= \; (p, v) \mid \varphi \lor \varphi \mid \varphi \land \varphi
  \mid EX^k \varphi \mid AX^K \varphi \mid \forall v : \phi,
\]
where  $p \in \atprop$, $v \in \atvar$ and $k \geq  1$.
\end{definition}

The formula $AX^k \varphi$ is read as ``for all paths, $\varphi$ holds at the
$k$-th step'' and the formula $EX^k \varphi$ stands for ``there is a path for
which $\varphi$ holds at the $k$-th step''. We inductively abbreviate bounded until as
\begin{align*}
 E(\varphi U^1 \psi) &\triangleq \psi \lor (\varphi \land EX^1 \psi) \\
 E(\varphi U^k \psi) &\triangleq \psi \lor (\varphi \land EX^1 (E (\varphi U^{k-1} \psi)).
\end{align*}
with the meaning ``there is a path in which $\psi$ holds at some point in the
$k$ following steps and before then $\varphi$ is true along the path'. The dual
until, prefixed by the universal operator, can be defined analogously.

A $\bictl$ formula is said to be a sentence if every variable appearing the
formula is in the scope of a universal quantifier. Hereafter we consider only
indexed $\bictl$ sentences of the form $\forall v_1 \ldots \forall v_m :
\varphi$, or, shortly, $\bictlspec$, where $v_i \in \atvar$ for $i \in
\set{1,\ldots,m}$ and $\varphi$ does not contain any universal quantifier.

We now define the satisfaction relation for $\bictl$ on the temporal models
associated with the concrete systems.

\begin{definition}[Satisfaction]
  \label{def:sat} 
  Given a model $\msys n$, a global state~$q^0$ and a $\bictl$ sentence
  $\bictlspec$ with $m \leq n$, the \emph{satisfaction} of $\varphi$ at $q$, denoted $(\msys n, q)
  \models\varphi$, or simply $q \models \varphi$ when $\msys n$ is clear from
  the context, is defined as follows:
  \begin{description}
  \item[$q \models (p, a)$] \ iff \ $q^o \in \valuation n((p, a))$, for $p \in
  \atprop$ and $a \in \agents n$; 
  \item[$q \models \varphi \lor \psi$] \ iff \ $q \models \varphi$ or $q \models
  \psi$;
  \item[$q \models \varphi \land \psi$] \ iff \ $q \models \varphi$ and $q
  \models \psi$;
  \item[$q \models EX^k  \varphi$] \ iff  there is $\rho \in \paths{q}$ such
  that $\rho(k) \models \varphi$;
  \item[$q \models AX^k  \varphi$] \ iff for all $\rho \in \paths{q}$ we
  have that $\rho(k) \models \varphi$;
  \item[$q \models \bictlspec$] \ iff for all $h : \set{v_1, \ldots, v_m}
  \rightarrow \set{\ag 1, \ldots,\ag n}$ we have that $q \models 
  \varphi[v_1 \mapsto h(v_1), \ldots v_m \mapsto h(v_m)]$,
 \end{description}
\end{definition}

A sentence $\bictlspec$ is said to be true in $\msys n$, denoted $\msys n
\models \bictlspec$ if $(\msys n, \globalinit n) \models \bictlspec$. The
sentence is said to be true in $\npis$, denoted $\npis \models \bictlspec$, if
$\bictlspec$ is true in every model induced by every concrete system
instantiated from $\npis$ with at least $m$ agents, i.e., $\forall n \geq m :
\msys n \models \bictlspec$.  The {\em parameterised verification problem} is to
check whether this holds.


\begin{definition}[Parameterised verification problem]
  Given an NPIS $\npis$ and a $\bictl$ sentence $\bictlspec$, determine whether
  $\npis \models \bictlspec$.
\end{definition}

  
  % Lomuscio, 2016b]. Given a set IND of indices, a set L AP
  % of local atomic propositions and a set G AP of global
  % atomic propositions, IACTLK\X formulae are defined by
  % the following BNF grammar:
  % φ ::= (p, v) | ¬(p, v) | q | ¬q | φ ∧ φ | φ ∨ φ | A(φU φ) |
  % A(φRφ) | Kiφ | ∀v : φ
  % where p ∈ L AP , q ∈ G AP , and v ∈ IND. The epis-
  % temic modality Kiφ is read as “agent i knows that φ”. The
  % temporal modality A(φU ψ) stands for “for all paths, at some
  % point ψ holds and before then φ is true along the path”; and
  % A(φRψ) denotes “for all paths, ψ holds along the path up to
  % and including the point when φ becomes true in the path”. An
  % IACTLK\X formula is said to be a sentence if every variable
  % appearing the formula is in the scope of a universal quantifier.
  % Hereafter we consider only indexed IACTLK\X sentences.
  % We now define the satisfaction relation.

  % We verify NIS against a bounded variant of a restricted
  % subset of ATL∗ (Alur, Henzinger, and Kupferman 2002),
  % drawing inspiration from Real Time Computation Tree
  % Logic (RTCTL) (Emerson et al. 1992). The satisfaction sta-
  % tus of the formulae expressible in our language depends only
  % on paths of a bounded length. This has been shown to be
  % practically relevant by enabling the efficient identification
  % of shallow bugs in a system’s execution (Biere et al. 2003;
  % Penczek, Wo´zna, and Zbrzezny 2002). In addition to this


  % practical consideration, our restriction to a bounded frag-
  % ment of ATL∗ follows the undecidability of the verification
  % problem for unbounded formulae (Akintunde et al. 2020a

\begin{example}
  We consider the example of a guarding game. In this game there is a colony of agents. The colony needs to be guarded by exactly one guard. Guarding duty costs the guard some health, while those who are resting improve their health. If no one is guarding the colony, then all the agents in the colony loose some health. When an agent does not have any health left, it ``dies''.

  We formalise this game as follows.

  $
  \begin{array}{rl}
    \lstates{t} = & \{(h,S) \mid h \in \mathbb{Z}, \ S \in \{G_s, R_s, E_s\} \\
    \acts{t} = & \{G_a, R_a, E_a\}\\

    \prot{t}(per) = & \left\{
                      \begin{array}{rl}
                        \{G_a, R_a\}, & \text{ if } h > G_p\\ 
                        \{R_a\}, & \text{ if } 0 < h \leq G_p\\ 
                        \{E_a\}, & \text{ if } h \leq 0\\
                      \end{array}
    \right.
  \end{array}
  $


  $
  \begin{array}{l@{~}l}
    \tr{t}((h,G_s), G_a, A) = & (h - G_p, G_s) \text{ if }G_a\notin A \text{ and } h > G_p\\
    \tr{t}((h,G_s), G_a, A) = & (0, E_s) \text{ if }G_a\notin A \text{ and } h \leq G_p\\
    \tr{t}((h,G_s), R_a, A) = & (h + R_r, R_s) \text{ if }G_a\in A\\
    \tr{t}((h,G_s), R_a, A) = & (h - U_p, R_s) \text{ if }G_a\notin A \text{ and } h > U_p\\
    \tr{t}((h,G_s), R_a, A) = & (0, E_s) \text{ if }G_a\notin A \text{ and } h \leq U_p\\[2mm]

    \tr{t}((h,G_s), G_a, A) = & (h - G_p, G_s) \text{ if }G_a\notin A \text{ and } h > G_p\\
    \tr{t}((h,G_s), G_a, A) = & (0, E_s) \text{ if }G_a\notin A \text{ and } h \leq G_p\\
    \tr{t}((h,R_s), R_a, A) = & (h + R_r, R_s) \text{ if }G_a\in A\\
    \tr{t}((h,G_s), R_a, A) = & (h - U_p, R_s) \text{ if }G_a\notin A \text{ and } h>U_p\\
    \tr{t}((h,G_s), R_a, A) = & (0, E_s) \text{ if }G_a\notin A \text{ and } h \leq U_p\\[2mm]
    
    \tr{t}((h,E_s), E_a, A) = & (h, E_s)\\

\end{array}
  $

  
\end{example}

%%% Local Variables:
%%% mode: latex
%%% TeX-master: "../main"
%%% End:
