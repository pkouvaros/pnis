In this section we put forward a procedure for solving the parameterised
verification problem introduced in the previous section. For ease of
presentation we fix a PNIS $\pnis = \tuple{t, e, \ell}$, where $\ag t =
\tuple{\lstates t, \init t, \obs t, \acts t, \prot t, \tr t }$, $\ag e =
\tuple{\lstates e, \acts e, \prot e, \tr e }$, and a $\bictl$ sentence
$\bictlspec$ throughout the section. The verification procedure that we
introduce recasts the parameterised verification problem for $\pnis$ and
$\bictlspec$  to a (standard) verification problem for NIS $\pnis_{zo}$, which
we below define as a zero-one abstraction of the concrete systems generated from
$\pnis$,  and the $\bctl$ formula $\varphi[v_1 \mapsto 1, \ldots, v_m \mapsto
m]$.  We show that the satisfaction status of $\varphi[v_1 \mapsto 1, \ldots,
v_m \mapsto m]$ on $\pnis_{zo}$ determines the the satisfaction status of
$\bictlspec$ on $\pnis$. This enables us to use previously established
methodologies for the verification of NIS against $\bctl$~\cite{Akintunde+20b}
to solve the verification problem for PNIS.


We start by reducing the problem of checking the $\bictl$ sentence in question
to that of checking a $\bctl$ formula. This formula is any ground instantiation
of $\bictlspec$ which we fix to $\varphi[v_1 \mapsto 1, \ldots, v_m \mapsto m]$.


\begin{lemma}[Symmetry reduction]
$\pnis \models \bictlspec$ iff $\pnis \models \varphi[v_1 \mapsto 1, \ldots, v_m
\mapsto m]$.
\end{lemma}
\begin{proof}
The lemma follows from the inherent symmetry present in systems comprising
homogeneous agents. The Appendix includes the full proof.
\end{proof}

We next construct the zero-one abstraction of the systems generated from
$\pnis$. The zero-one abstraction is a NIS comprising a zero-one agent, which is
an abstraction for arbitrarily many concrete agents, $m$ concrete agents, whose
local states determine the satisfaction status of the atomic propositions in
$\varphi[v_1 \mapsto 1, \ldots, v_m \mapsto m]$ (see
Definition~\ref{def:concreteystem}), and the environment. In other words, the
zero-one agent defined below encodes how an arbitrary number of agents may
interfere with the temporal evolution of the concrete agents $1, \ldots, m$.

\begin{definition}[Zero-one agent]
Given an agent template $\ag t = \tuple{\lstates t, \init t, \obs t,  \acts t,
\prot t, \tr t }$ over a set $\pers t$ of percepts and a set $\prvs t$ of
private states, its associated \emph{zero-one agent} is a tuple $\ag{zo} =
\tuple{\lstates{zo}, \init{zo}, \obs{zo}, \acts{zo}, \prot{zo}, \tr{zo}}$ over
sets $\pers{zo} = 2^{\pers t} \setminus \set{\emptyset}$ and $\prvs{zo}=
2^{\prvs t} \setminus \set{\emptyset}$ of percepts and private states, where:
\begin{itemize} 
    \item $\lstates{zo} = 2^{\lstates t} \setminus \set{\emptyset}$ is the set
    of abstract states. An abstract state represents the projection of global
    states in systems of any size onto a set.

    \item $\init{zo} = \set{\init t}$ is the unique initial abstract state.
    
    \item $\obs{zo} : \lstates{zo} \times \lstates e \rightarrow \per{zo}$ is
    abstract observation function. It maps pairs of abstract and environment
    states to sets of percepts, where each set includes the percepts that would
    be collectively generated in a global state represented by the abstract
    state. Formally, the observation function satisfies $\obs{zo}(l_{zo}, l_e) =
    \per{zo}$ if:
    \begin{itemize}
        \item for all $l_t \in l_{zo}$ we have that $\obs t(l_t, l_e) \in
        \per{zo}$;
        \item for all $\per t \in \per{zo}$ there is $l_t \in
        l_{zo}$ such that $\obs t(l_t, l_e) = \per t$.
    \end{itemize}

    \item $\acts{zo} = 2^{\lstates t \times \acts t} \setminus \set{\emptyset}$
    is the set of abstract actions. Analogously to abstract states, an abstract
    action represents the projection of joint actions, paired with the local
    states at which they are performed,  of arbitrarily many agents onto a set.

    \item $\prot{zo} : \lstates{zo} \rightarrow 2^{\acts{zo}} \setminus
    \set{\emptyset}$ is the abstract protocol. The protocol prescribes the sets
    of template actions that can be collectively performed at a global state
    represented by a given abstract state. It is defined as $$\prot{zo}(l_{zo})
    = \set{ \bigcup_{l_t \in l_{zo}} \set{l_t} \times A_{l_t} \mid A_{l_t} \in
    2^{\prot t(l_t)} \setminus \set{\emptyset}}.$$

    \item $\tr{zo} \colon \lstates{zo} \times \acts{zo} \times 2^{\acts t} 
    \times \acts e \rightarrow \prvs{zo}$ is the abstract transition function.
    The function determines the set of private states that the agents would
    collectively transition to in any global state represented by a given
    abstract state after they have performed a joint action represented by a
    given abstract action. It is s.t. $\tr{zo}(l_{zo}, \alpha_{zo}, A, \alpha_e)
    = \prv{zo}$ if the following hold:
    \begin{itemize}
        \item $\alpha_{zo} \in \prot{zo}(l_{zo})$; 
        \item for all $(l_t, \alpha_t) \in \alpha_{zo}$ we have that $\tr t(l_t,
        \alpha_t, A', \alpha_e) \in \prv{zo}$, where $A' = A \cup \set{\alpha'_t
        \mid (l'_t, \alpha'_t) \in \alpha_{zo} \text{ for some } l'_t})$; 
        \item for all $\prv t \in \prv{zo}$, there is $(l_t, \alpha_t) \in
        \alpha_{zo}$ s.t. $\tr t(l_t, \alpha_t, A', \alpha_e) = \prv t$, where
        $A'$ is as in the above clause.  
    \end{itemize}
\end{itemize}
\end{definition}

The zero-one NIS is a tuple $\pnis_{zo} = \tuple{\set{1,\ldots,m,zo,e},
\globalinit{m}_{zo}, \valuation{m}_{zo}}$, where $\globalinit{m}_{zo} =
\tuple{\init 1, \ldots, \init m, \init{zo}, \init e}$ is the initial global
state and $\valuation{m}_{zo} : \atprop \times \set{1, \ldots, m } \rightarrow
2^{\globalstates{m}_{zo}}$ is the concrete labelling function satisfying $q \in
\valuation{m}_{zo}(p,a)$ iff $\ls{a}{q} \in \tlabel a(p)$.  The global
transition function is defined as in Definition~\ref{def:globaltransition} but
replacing $A$ in the environment conditions in the first clause with $A =
\set{\la{a}{q} \mid a \in \set{1,\ldots,n}} \cup \la{zo}{q}$, similarly
replacing~$A$ for the concrete agents' conditions in the second clause, and
adding analogous conditions for the zero-one agent. Given the global transition
function we can similarly associate a model  $\masys m = \tuple{\aglobalstates
m}$

\begin{theorem}
$\masys m \models \varphi[v_1 \mapsto 1, \ldots, v_m \mapsto m]$ iff $\pnis
\models  \varphi[v_1 \mapsto 1, \ldots, v_m \mapsto m]$.
\end{theorem}
\begin{proof}
The left to right direction follows from the property of the zero-one model to
simulate every concrete with at least $m+1$ agents.  For the right to left
direction  assume that $\masys m \not \models \varphi[v_1 \mapsto 1, \ldots, v_m
\mapsto m]$l. Then, $\masys m  \models \neg \varphi[v_1 \mapsto 1, \ldots, v_m
\mapsto m]$. As $\neg \varphi[v_1 \mapsto 1, \ldots, v_m \mapsto m]$ includes
only existential path quantification, there is a path evidencing the
satisfaction of the formula. The theorem can be proved by induction
constructing a similar concrete path that also evidences the satisfaction of
said formula. This implies that $\varphi[v_1 \mapsto 1, \ldots, v_m \mapsto m]$
is not satisfied by the underlying concrete system and thus it is neither
satisfied by the parameterised system. The Appendix includes the full proof.
\end{proof}


%%% Local Variables:
%%% mode: latex
%%% fill-column: 79
%%% TeX-master: "../main"
%%% End:
