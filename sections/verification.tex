In this section we put forward  procedures for solving the parameterised
verification and emergence identification problems introduced in the previous
section. For ease of presentation we fix a PNIS $\pnis = \tuple{t, e, \ell}$,
where $\ag t = \tuple{\lstates t, \init t, \obs t, \acts t, \prot t, \tr t }$,
$\ag e = \tuple{\lstates e, \acts e, \prot e, \tr e }$, and a $\bictl$ sentence
$\bictlspec$ throughout the section. The verification procedure that we
introduce recasts the parameterised verification  and emergence identification
problems for $\pnis$ and $\bictlspec$  to a number of  (standard) verification
problems for  abstract and concrete NIS  against the $\bctl$ formula
$\varphi[v_1 \mapsto 1, \ldots, v_m \mapsto m]$.  We show that the satisfaction
status of $\varphi[v_1 \mapsto 1, \ldots, v_m \mapsto m]$ on 
these systems determines the the satisfaction and emergent status of
$\bictlspec$ on $\pnis$. This enables us to use previously established
methodologies for the verification of NIS against $\bctl$~\cite{Akintunde+20b}
to analyse PNIS.


We start by reducing the problem of checking the $\bictl$ sentence in
question to that of checking a $\bctl$ formula that is a ground
instantiation of $\bictlspec$.  We fix it to
$\varphi[v_1 \mapsto 1, \ldots, v_m \mapsto m]$ and for brevity denote
as $\varphi[m]$.


\begin{lemma}[Symmetry reduction]
\label{lemma:symmetry}
$\pnis \models \bictlspec$ iff $\pnis \models \varphi[m]$.
\end{lemma}
\begin{proof}
The lemma follows from the inherent symmetry present in systems comprising
homogeneous agents. The Appendix includes the full proof.
\end{proof}

We next construct the zero-one abstraction of the systems generated from
$\pnis$. The zero-one abstraction is a NIS comprising a zero-one agent, which is
an abstraction for arbitrarily many concrete agents, $m$ concrete agents, whose
local states determines the satisfaction status of the atomic propositions in
$\varphi[m]$ (see
Definition~\ref{def:concreteystem}), and the environment. In other words, the
zero-one agent defined below encodes how an arbitrary number of agents may
interfere with the temporal evolution of the concrete agents $1, \ldots, m$.

\begin{definition}[Zero-one agent]
Given an agent template $\ag t = \tuple{\lstates t, \init t, \obs t,  \acts t,
\prot t, \tr t }$ over a set $\pers t$ of percepts and a set $\prvs t$ of
private states, its associated \emph{zero-one agent} is a tuple $\ag{zo} =
\tuple{\lstates{zo}, \init{zo}, \obs{zo}, \acts{zo}, \prot{zo}, \tr{zo}}$ over
sets $\pers{zo} = 2^{\pers t} \setminus \set{\emptyset}$ and $\prvs{zo}=
2^{\prvs t} \setminus \set{\emptyset}$ of percepts and private states, where:
\begin{itemize} 
    \item $\lstates{zo} = 2^{\lstates t} \setminus \set{\emptyset}$ is the set
    of abstract states. An abstract state represents the projection of global
    states in systems of any size onto a set.

    \item $\init{zo} = \set{\init t}$ is the unique initial abstract state.
    
    \item $\obs{zo} : \lstates{zo} \times \lstates e \rightarrow \per{zo}$ is
    abstract observation function. It maps pairs of abstract and environment
    states to sets of percepts, where each set includes the percepts that would
    be collectively generated in a global state represented by the abstract
    state. Formally, the observation function satisfies $\obs{zo}(l_{zo}, l_e) =
    \per{zo}$ if:
    \begin{itemize}
        \item for all $l_t \in l_{zo}$ we have that $\obs t(l_t, l_e) \in
        \per{zo}$;
        \item for all $\per t \in \per{zo}$ there is $l_t \in
        l_{zo}$ such that $\obs t(l_t, l_e) = \per t$.
    \end{itemize}

    \item $\acts{zo} = 2^{\lstates t \times \acts t} \setminus \set{\emptyset}$
    is the set of abstract actions. Analogously to abstract states, an abstract
    action represents the projection of joint actions, paired with the local
    states at which they are performed,  of arbitrarily many agents onto a set.

    \item $\prot{zo} : \lstates{zo} \rightarrow 2^{\acts{zo}} \setminus
    \set{\emptyset}$ is the abstract protocol. The protocol prescribes the sets
    of template actions that can be collectively performed at a global state
    represented by a given abstract state. It is defined as $\prot{zo}(l_{zo})
    = \set{ \set{l_t \times A_{l_t} \mid l_t \in l_{zo}} \mid A_{l_t} \in
    2^{\prot t(l_t)} \setminus \set{\emptyset}}.$

    \item $\tr{zo} \colon \lstates{zo} \times \acts{zo} \times 2^{\acts t} 
    \times \acts e \rightarrow \prvs{zo}$ is the abstract transition function.
    The function determines the set of private states that the agents would
    collectively transition to in any global state represented by a given
    abstract state and after they have performed a joint action represented by a
    given abstract action. It is such that $\tr{zo}(l_{zo}, \alpha_{zo}, A, \alpha_e)
    = \prv{zo}$ if the following hold:
    \begin{itemize}
        \item $\alpha_{zo} \in \prot{zo}(l_{zo})$; 
        \item for all $(l_t, \alpha_t) \in \alpha_{zo}$ we have that $\tr t(l_t,
        \alpha_t, A', \alpha_e) \in \prv{zo}$, where $A' = A \cup \set{\alpha'_t
        \mid (l'_t, \alpha'_t) \in \alpha_{zo} \text{ for some } l'_t \in \lstates{t}})$;
        \item for all $\prv t \in \prv{zo}$, there is $(l_t, \alpha_t) \in
        \alpha_{zo}$ s.t. $\tr t(l_t, \alpha_t, A', \alpha_e) = \prv t$, where
        $A'$ is as in the above clause.  
    \end{itemize}
\end{itemize}
\end{definition}

The abstract  NIS comprises the zero-one agent and~$m$ concrete agents. It is a
tuple $\pnis_{ab} = \tuple{\set{1,\ldots,m,zo,e}, \globalinit{m}_{ab},
\valuation{m}_{ab}}$, where $\globalinit{m}_{ab} = \tuple{\init 1, \ldots, \init
m, \init{ab}, \init e}$ is the initial global state and $\valuation{m}_{ab} :
\atprop \times \set{1, \ldots, m } \rightarrow 2^{\globalstates{m}_{ab}}$ is the
concrete labelling function satisfying $q \in \valuation{m}_{ab}(p,i)$ iff
$\ls{i}{q} \in \tlabel i(p)$.  The global transition function is defined as in
Definition~\ref{def:globaltransition} but replacing $\setprojection{\alpha}$ in
the environment and concrete agents' clauses $\set{\la{a}{q} \mid a \in
\set{1,\ldots,n}} \cup \la{zo}{q}$,  and adding analogous conditions for the
zero-one agent (see the Appendix for the full definition). Given the global
transition function we can similarly associate a (abstract) model  $\masys m =
\tuple{\aglobalstates m, \aglobalacts m, \aglobalrel m, \avaluation m}$.

We now establish a correspondence between the abstract model and the concrete
models. We show in particular that (i) the abstract model simulates every
concrete model with at least~$m+1$ agents; (ii) there is always a concrete model
with a sufficient number of agents that simulates the abstract model; and (iii)
a concrete model always simulates a smaller concrete model. A model simulates another
model if every behaviour exhibited by the latter is also admitted by the former.
As specifications for PNIS are bounded, we consider simulation up to a bounded
number of steps as defined below.

\begin{definition}[Bounded simulation.]
A $b$-bounded simulation between two models $\mathcal M = \tuple{G, \it{ACT}, T,
\ell}$ and $\mathcal{M'} = \tuple{G', \it{ACT'}, T', \ell'}$  with initial
global states $\iota$ and $\iota'$ is inductively defined on $b \geq 0$ as
follows.
\begin{itemize}
    \item A relation $\sim_0 \subseteq G \times G'$ is  $0$-bounded simulation
    if $(\iota, \iota') \in \sim_0$ and whenever $(q, q') \in \sim_0$, we have
    that $q \in \ell(p,i)$ implies that $q' \in \ell'(p, i)$.
    \item A relation $\sim_b \subseteq G \times G'$ is  $b$-bounded simulation
    if $(\iota, \iota') \in \sim_b$ and  whenever $(q, q') \in \sim_b$, the
    following conditions hold:
    \begin{enumerate}
    \item $(q, q') \in \sim_0$.
    \item If $(q, \alpha, q^1) \in T$ for a joint action $\alpha \in \it{ACT}$ and
    global state $q^1 \in G$, then there is a joint action
    $\alpha' \in \it{ACT}'$ and global state $g'^1 \in G'$ such that
    $(g', \alpha', g'^1) \in T'$ and 
    $(q^1, q'^1) \in \sim_{b - 1}$.  
\end{enumerate}
\end{itemize}
\end{definition}

We say that a model $\mathcal{M'}$ simulates a model $\mathcal M$ up to~$b$ time
steps, denoted $\mathcal M \leq_b \mathcal{M'}$, if there is a $b$-bounded
simulation relation between $\mathcal M$ and $\mathcal{M'}$.  Universal $\bctl$
formulae are preserved from the simulating model to the simulated model and
existential $\bctl$ formulae are preserved from the simulated model to the
simulating model whenever their temporal depth is at most~$b$. 

\begin{theorem} Let $\mathcal M$ and $\mathcal M'$ be two models such $\mathcal
M \sim_b  \mathcal M'$. Then, the following hold.
\label{th:sim}
\begin{enumerate}
    \item If $\mathcal M' \models \varphi$ for a universal $\bctl$ formula
    $\varphi$ with $\td{\varphi} \leq b$, then $\mathcal M \models
    \varphi$.
    \item If $\mathcal M \models \varphi$ for an existential $\bctl$ formula
    $\varphi$ with $\td{\varphi} \leq b$, then $\mathcal M' \models
    \varphi$.
\end{enumerate}
\end{theorem}
\begin{proof}[Proof Sketch.]
The result is established in~\cite{ClarkeGrumbergLong94} for arbitrary Kripke
structures and unbounded $\ctl$ formulae and simulations. Since $\bctl$ formulae
are evaluated up to bounded temporal depths, they are preserved under bounded
simulations.
\end{proof}

We can now show the simulation results pertaining to the abstract and
concrete model.  We start by showing that the abstract model simulates every
concrete model with at least~$m+1$ agents.

\begin{theorem}
\label{th:ab-concr-sim}
$\msys n \leq_b \masys m$ for any $n \geq m+1$ and $b \geq 0$.
\end{theorem}
\begin{proof}
The proof proceeds by induction on~$b$ and it is included in the Appendix.
\end{proof}

Irrespective of the temporal depth of the specification under
analysis there is always a concrete model that simulates the abstract model up
to that depth.

\begin{theorem}
\label{th:concr-ab-sim}
There is $n \geq m+1$ such that $\masys m \leq_b \msys n$ for any $b \geq 0$.
\end{theorem}
\begin{proof}
The proof proceeds by induction on~$b$ and it is included in the Appendix.
\end{proof}


Every concrete model simulates every smaller concrete model.

\begin{theorem}
\label{th:concr-sim}
For every $n' > n$ we have that $\msys n \leq_b \msys{n'}$ for any
$b \geq 0$.
\end{theorem}
\begin{proof}
The proof proceeds by having every additional agent in $\msys{n'}$ to mimic
agent~1 in $\msys n$ and it is included in the Appendix.
\end{proof}

The above results enable the derivation of procedures for solving
the parameterised verification and emergence identification problems. In the case
of universal formulae the emergence identification procedure simply concerns
checking the abstract model and the parameterised verification procedure 
additionally involves checking a single concrete model against the formula in
question.

\begin{corollary}
\label{cor:universal}
If $\varphi[m]$ is a universal $\bctl$ formula and we have that $\masys m
\models  \varphi[m]$, then the following hold:
\begin{itemize}
    \item  If $\masys m \models \varphi[m]$, then $m + 1$ is an emergence
    threshold for $\bictlspec$. Otherwise, there is no emergence threshold for
    $\bictlspec$.
    \item $\masys m \models \varphi[m] \land \msys m \models \varphi[m]$ iff
    $\pnis \models \bictlspec$.
\end{itemize}
\end{corollary}
\begin{proof}
To prove the first clause assume that $\masys m \models \varphi[m]$. By
Theorem~\ref{th:ab-concr-sim} we have that $\msys n \models \varphi[m]$ for
every $n \geq m+1$. Lemma~\ref{lemma:symmetry} gives that  $\msys n \models
\bictlspec$ for every $n \geq m+1$. Hence, $m+1$ is an emergence threshold for
$\bictlspec$. If otherwise we have that  $\masys m \not \models \varphi[m]$,
then by Theorem~\ref{th:concr-ab-sim} there is $n \geq m+1$ such that $\msys n
\not \models \varphi[m]$. Theorem~\ref{th:concr-sim} additionally gives that
$\msys{n'} \not \models \varphi[m]$ for every $n' \geq n$. Thus, from
Lemma~\ref{lemma:symmetry}, we obtain that $\msys n' \not \models \bictlspec$
for every $n' \geq n$, therefore there is no emergence threshold for
$\bictlspec$.

To prove the second clause assume that $\masys m \models \varphi[m]$ and $\msys
m \models \varphi[m]$. By Theorem~\ref{th:ab-concr-sim} we have that $\msys n
\models \varphi[m]$ for every $n \geq m$. From Lemma~\ref{lemma:symmetry} we
obtain that $\msys n \models \bictlspec$ for every $n \geq m$, thus $\pnis
\models \bictlspec$. If otherwise $\masys m \not \models \varphi[m]$ or $\msys m
\not \models \varphi[m]$, then by Theorem~\ref{th:concr-ab-sim}, there is $n
\geq m$ such that $\msys n \not \models \varphi[m]$.  From
Lemma~\ref{lemma:symmetry} we obtain that there is $n \geq m$ such that $\msys n
\not \models \bictlspec$. Consequently, $\pnis \not \models \bictlspec$.
\end{proof}

% So every universal $\bctl$ formula satisfied by the abstract model
% and the concrete model of $m$ agents is also satisfied by the parameterised
% system. This also implies that~$m$ is an emergence threshold for the formula.

Theorems~\ref{th:ab-concr-sim},~\ref{th:concr-ab-sim} and
~\ref{th:concr-sim},additionally enable the derivation of  simple procedure
for the verification of existential properties.

\begin{corollary}
\label{cor:existential}
If $\varphi[m]$ is an existential $\bictl$ formula with temporal depth $b$ and  $n
= \min_i \left( \masys m \leq_b \msys i \right)$, then the following hold:
\begin{itemize}
\item If $\msys n \models \varphi[m]$, then $n$ is an emergence threshold for
$\bictlspec$. Otherwise, there is no emergence threshold for $\bictlspec$.

\item $\msys i \models \bictlspec$, for all $i \in \set{m, \ldots, n}$, iff
$\pnis \models \bictlspec$.
\end{itemize}
\end{corollary}
\begin{proof}
To prove the first clause assume that $\msys n \models \varphi[m]$. By
Theorem~\ref{th:concr-sim} we have that $\msys{n'} \models \varphi[m]$ for every
$n' \geq n$. Lemma~\ref{lemma:symmetry} gives that  $\msys{n'} \models
\bictlspec$ for every $n' \geq n$. Hence, $n$ is an emergence threshold for
$\bictlspec$. If otherwise we have that  $\msys n \not \models \varphi[m]$, then
by Theorem~\ref{th:concr-ab-sim} it follows  that $\masys m \not \models
\varphi[m]$, hence, by Theorem~\ref{th:ab-concr-sim}, $\msys{n'} \not \models
\varphi[m]$ for every $n' \geq m+1$.  Thus, from Lemma~\ref{lemma:symmetry}, we
obtain that $\msys{n'} \not \models \bictlspec$ for every $n' \geq m + 1$,
therefore there is no emergence threshold for $\bictlspec$.

To prove the second clause assume that $\msys i \models \varphi[m]$ for every $i
\in \set{m,\ldots,n}$.  By Theorem~\ref{th:concr-sim} we have that $\msys{i}
\models \varphi[m]$ for every $i \geq m$. Lemma~\ref{lemma:symmetry} gives that
$\msys{i} \models \bictlspec$ for every $i \geq m$. Hence, $\pnis \models
\bictlspec$.  If otherwise there is $i \in \set{m, \ldots, n}$ such that $\msys
i \not \models \varphi[i]$, then, by Lemma~\ref{lemma:symmetry}, we have that
$\msys i \not \models \bictlspec$, which trivially implies that $\pnis \not
\models \bictlspec$.
\end{proof}
% \begin{proof}[Proof Sketch]
%     The proof is analogous to the proof of Corollary\ref{cor:universal}.
% \end{proof}

In summary, Corollaries~\ref{cor:universal} and~\ref{cor:existential} provide
constructive, sound and complete methodologies for checking universal and 
existential specifications for PNIS.  For universal properties, verification can
be conducted by constructing and checking the abstract model and the concrete
model with~$m$ agents, where $m$ is the the number of index variables in the
specification in question. The specification is satisfied by the abstract and
concrete models iff the specification is satisfied in general for any number of
agents.  The satisfaction of the specification by the abstract model is also
connected by biconditional implication with the existence of an emergent
threshold for the specification. For existential properties, verification can be
performed by enumerating all concrete models, identifying the smallest one that simulates the abstract model,
 and checking all concrete models up to the simulating one. 
 The specification is satisfied by all these
concrete models iff the specification is satisfied in general for any number of
agents.   The satisfaction of the specification by the concrete model that
simulates the abstract model is also connected by biconditional implication with
the existence of an emergent threshold for the specification. 



% \begin{corollary}
% If $\bictlspec$ is an existential $\bictl$ sentence with temporal depth $b$, $n
% = \min_i \left( \masys m \leq_b \msys i \right)$, and we have that $\msys n
% \models \bictlspec$, then $n$ is an emergence threshold for $\bictlspec$.
% \end{corollary}



% It follows that if the abstract model does not satisfy the formula in question
% then the parameterised system does not satisfy the formula either. Moreover,
% since the abstract model also simulates said concrete model and as the
% simulation relation is transitive, it follows that the concrete model simulates
% every bigger model, hence there is no emergence threshold for the specification.


% \begin{corollary}
% If $\varphi[m]$ is a universal $\bctl$ formula and we have that $\masys m \not
% \models  \varphi[m]$, then it follows that (i) $\pnis \not \models \bictlspec$
% and (ii) there is no emergence threshold for $\bictlspec$.
% \end{corollary}
% \begin{proof}
% The result follows from Theorem~\ref{th:ab-sim}, Theorem~\ref{th:concr-sim} and
% Lemma~\ref{lemma:symmetry}.
% \end{proof}

%%% Local Variables:
%%% mode: latex
%%% fill-column: 79
%%% TeX-master: "../main"
%%% End:
